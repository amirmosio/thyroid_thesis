% -------------------------------------------------------
%  Abstract
% -------------------------------------------------------


\pagestyle{empty}

\شروع{وسط‌چین}
\مهم{چکیده}
\پایان{وسط‌چین}
\بدون‌تورفتگی

سرطان تیررویید یکی از انواع شایع سرطان در بدن انسان‌هاست که در آن غده تیروئید درگیر می شود که انواع مختلفی نیز دارد.
یکی از راه‌های اصلی تشخیص این بیماری، استفاده از اسلاید‌های دیجیتال ثبت شده از نمونه ای از سلول‌هاست.
به این صورت که متخصصان با استفاده از ویژگی‌هایی که سلول‌های سرطانی به خود می گیرند می توانند آن‌ها را از سلول‌های عادی تیروئید تشخصی دهند و مقدار درگیری نمونه به سرطان را تشخیص دهند.
ما هم در این پروژه سعی داریم تا با استفاده از روش‌های مبتنی بر یادگیری ماشین و پردازش تصویر، مدل خود را به گونه ای آموزش دهیم تا با استفاده از این ویژگی‌های متفاوت سلول‌های سرطانی از روی اسلاید‌ها، تشخیصی با دقت بالا انجام دهیم تا بتواند به کمک متخصصان این زمینه بیاید.

\پرش‌بلند
\بدون‌تورفتگی \مهم{کلیدواژه‌ها}:
یادگیری ماشین، پردازش تصویر، سرطان تیروِیید، بدخیم، اسلاید‌های دیجیتال
\صفحه‌جدید
