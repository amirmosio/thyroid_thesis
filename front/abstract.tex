% -------------------------------------------------------
%  Abstract
% -------------------------------------------------------


\pagestyle{empty}

\شروع{وسط‌چین}
\مهم{چکیده}
\پایان{وسط‌چین}
\بدون‌تورفتگی

سرطان تیروئید یکی از انواع شایع سرطان در بدن انسان‌هاست که در آن غده تیروئید درگیر می‌شود.
یکی از راه‌های اصلی تشخیص این بیماری، استفاده از اسلاید‌های دیجیتال ثبت‌شده از نمونه‌‌های سلولی است.
به این صورت که متخصصان با استفاده از ویژگی‌هایی که سلول‌های سرطانی به خود می‌گیرند، می‌توانند آن‌ها را از سلول‌های عادی تیروئید تشخیص دهند و مقدار درگیری نمونه به سرطان را تشخیص دهند.
هدف از این پروژه این است که با به‌کارگیری روش‌های مبتنی بر یادگیری ماشین و پردازش تصویر، مدلی آموزش دهیم که با استفاده از اسلاید‌ها، تشخیصی با دقت بالا انجام دهد. از آنجایی که تشخیص این سرطان به این روش توسط نیروی متخصص بسیار زمان‌بر است، این مدل می‌تواند دستیار و کمک بسیار خوبی برای آن‌ها باشد.

\پرش‌بلند
\بدون‌تورفتگی \مهم{کلیدواژه‌ها}:
یادگیری ماشین، پردازش تصویر، سرطان تیروِیید، بدخیم، اسلاید‌های دیجیتال
\صفحه‌جدید
