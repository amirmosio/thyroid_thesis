% -------------------------------------------------------
%  Abstract
% -------------------------------------------------------


\pagestyle{empty}

\شروع{وسط‌چین}
\مهم{چکیده}
\پایان{وسط‌چین}
\بدون‌تورفتگی

سرطان تیروئید یکی از انواع شایع سرطان در بدن انسان‌هاست که در آن غده تیروئید درگیر می‌شود.
یکی از راه‌های اصلی تشخیص این بیماری، استفاده از اسلاید‌های دیجیتال ثبت‌شده از نمونه‌‌های سلولی است.
به این صورت که متخصصان با استفاده از ویژگی‌هایی که سلول‌های سرطانی به خود می‌گیرند، می‌توانند آن‌ها را از سلول‌های عادی تیروئید تشخیص دهند و مقدار درگیری نمونه به سرطان را تشخیص دهند.
هدف از این پروژه این است که با به‌کارگیری معماری‌های \lr{ResNet} و \lr{Inception} مبتنی شبکه‌های عصبی عمیق پیچشی، مدلی آموزش دهیم که با استفاده از اسلاید‌ها، تشخیصی با دقت بالا انجام دهد. از آنجایی که تشخیص این سرطان به این روش توسط نیروی متخصص بسیار زمان‌بر است، این مدل می‌تواند دستیار و کمک بسیار خوبی برای آن‌ها باشد.
در آخر، نتایج ارزیابی بدست آمده از آموزش مدل‌ها، دقت $94.98$ را برای طبقه بند بدخیم-خوش‌خیم و دقت $99.72$ را برای طبقه‌بند سالم-توموری نشان می‌دهند که گویای ظرفیت این روش‌ها برای حل مسائل این چنینی است. 

\پرش‌بلند
\بدون‌تورفتگی \مهم{کلیدواژه‌ها}:
یادگیری ماشین، پردازش تصویر، سرطان تیروِیید، بدخیم، اسلاید‌های دیجیتال، \lr{ResNet}، \lr{Inception}
\صفحه‌جدید
