\فصل{روش‌های پیشنهادی}

در این پژوهش سعی شده با استفاده از این روش و با استفاده از اسلاید‌های دیجیتال ثبت‌شده از بیمار که شامل سلول‌های غده تیروئید هستند، روشی برای تشخیص سریع تر و دقیق تر سرطان تیروئید ارائه شود.
نمودار جریان داده را در شکل \ref{thesis_method_flowchart} آمده است.
\begin{figure}
	\begin{center}
		\vspace{0.5cm}
		\includegraphics[width=\linewidth]{figs/suggested_methods/chart.PNG}
	\end{center}
	\caption{نمودارجریان داده}
	\label{thesis_method_flowchart}
\end{figure}

\section{مجموع‌داده}\label{sec:مجموع‌داده}
یکی از چالش‌هایی که در حل این مسئله با آن روبرو هستیم، نبود مجموع‌داده مناسب و با دسترسی عمومی است و گام اول برای این پژوهش به حساب می‌آید.
در سال 2010 مقاله \cite{halicek2019head} منتشر شد که هدف آن تشخیص سرطان با به‌کارگیری اسلاید‌های ناحیه‌های سر و گردن بوده است.
این پژوهش، بعد از آموزش مدل بر روی مجموع‌داده خصوصی تهیه شده، سعی شد تا برای ارزیابی خارجی، مدل را بر روی مجموع‌داده عمومی \lr{CAMELYON 2016} تست کنند. این مجموع‌داده در چالشی در سال 2016 منتشر شد که شرکت کنندگان باید از اآن برای حل مسئله استفاده می‌کردند. هدف اصلی این چالش تشخیص سرطان سینه بوده و مجموع‌داده نیز شامل اسلاید‌ها و برچسب‌هایی مربوط به این سرطان بوده است.
در مقاله‌ای دیگر \cite{bohland2021machine} مجموع‌داده عمومی از اسلاید‌های دیجیتال و برچسب‌های آن تهیه شد. هدف آن تشخیص پاپیلاری بودن یا نبودن غده تیروئید بوده‌است که علیرغم اعلام نویسنده، موفق به تماس برای دریافت مجموع‌داده نشدیم.
با توجه به گفته‌های اخیر، جمع آوری داده یکی از بخش‌های مهم این پروژه است تا با آن بتوان مدل را به خوبی آموزش داد و عملکرد خوبی را دریافت کرد.
در ادامه به مجموع‌داده‌هایی که تهیه شد و درنهایت برای آموزش و ارزیابی مدل به‌کار گرفته شد اشاره می‌کنیم.

\subsection{اطلس ژنوم سرطان تیروئید}\label{subsec:موسسه-ملی-سرطان-پورتال-داده-های-مشترک-ژنومیک}

اطلس ژنوم سرطان تیروئید\LTRfootnote{The Cancer Genome Atlas-Thyroid Cancer/TCGA-THCA} بخشی از یک تلاش بزرگتر برای ایجاد یک جامعه تحقیقاتی متمرکز بر اتصال فنوتیپ‌های\LTRfootnote{Phenotypes} سرطان به ژنوتیپ‌ها\LTRfootnote{Genotypes} با ارائه تصاویر بالینی منطبق با افراد از اطلس ژنوم سرطان\cite{clark2013cancer} (TCGA) است.
%\lr{Cite: "The results <published or shown> here are in whole or part based upon data generated by the TCGA Research Network: %https://www.cancer.gov/tcga."}
این پروژه شامل اطلاعات 507 بیمار است که هدف اصلی آن، بررسی سرطان تیروئید در این بیماران است.
این اطلاعات شامل توالی دی ان ای، جنسیت، سن، داده‌های بالینی از بررسی بیمار توسط پزشک، اسلایدهای دیجیتال نمونه برداری‌شده از غده، درصد سلول‌های توموری و سالم و بسیاری موارد دیگر است.
تعداد کل اسلاید‌های جمع آوری‌شده مورد استفاده ما از این پروژه 639 عدد است که در مجموع حجمی بالغ بر 105 گیگابایت دارند. برای هرکدام از این اسلاید‌ها سه برچسب درصد سلول‌های عادی، درصد سلول‌های سرطانی و درصدر سلول‌ها استورمال آمده است که ما نیز از این برچسب‌ها برای مسئله خود استفاده می‌کنیم. از بین این اسلاید‌ها و با توجه به درصد سلول‌ها نرمال، توموری و استورمال، تعداد کل 182 اسلاید انتخاب شد که در بین آن‌ها، 106 اسلاید صددرصد توموری و 76 اسلاید آن، صددرصد عادی است.
%در مرحله بعد برای استفاده از این 182 اسلاید، آن‌ها، به 265418 قطعه عکس کوچک تر در آورده شدند تا بعد از آن بتوانند در آموزش مدل مورد %استفاده قرار بگیرند.
نمونه‌ای از اسلاید این مجموع‌داده در \ref{fig:sampleWSIscan}  و در فصل قبل آمده است.
در صورت نیاز، جزئیات بیشتری از این مجموع‌داده در آدرس \cite{ncigdc} قرار دارد.
\section{پاپ سوسایتی اطلس تیروئید}\label{subsec:پاپ-سوسایتی}
پاپ سوسایتی، انجمنی است که به عنوان یک شرکت اداره می‌شود و منحصراً به عنوان یک مؤسسه خیریه معاف از مالیات برای سازمان‌های علمی آموزشی، مدنی و خیریه سازماندهی شده است.
اطلس تیروئید این انجمن\cite{papsocietyiamgeatlas}، شامل نزدیک به 300 تصویر از غده تیروئید و نواحی مختلف آن از اسمیر مستقیم یک نمونه برداری سوزنی است که تمامی تصاویر به طور مستقل توسط متخصصین بررسی و برچسب گذاری شده‌اند.
تصاویر توسط اسکریپتی از صفحات وب این اطلس جمع آوری شد و که هر کدام از تصاویر ابعادی بین
 $1752\times1485$
 و
 $525\times360$
 را دارا هستند.
مجموع تصاویر این مجموع‌داده حجمی $125$ ماگابایتی دارند که از بین این تصاویر، $269$ عکس به عنوان، بدخیم و خوشخیم بچسب زده شده‌اند.
نمونه‌هایی از تصاویر این مجموع‌داده در \ref{fig:samplecoloringfrompapsociety}  و در فصل قبل آمده است.
%در انتها، از تصاویر این دو گروه، $396$ تصویر کوچک تر برای استفاده استخراج شد.
\subsection{پایگاه داده ریزآرایه بافت استنفورد}\label{subsec:پایگاه-داده-ریزآرایه-بافت-استنفورد}
پایگاه داده ریز آرایه بافت استنفور\cite{marinelli2007stanford} بخش تیروئید، شامل حدود 200 تصویر است که تصاویر و برچسب های آن توسط اسکریپتی از آدرس \cite{stanfortissuemicroarray} استخراج شدند. حجم این دیتاست 352 مگابایت است که در نهایت از گروه تصاویر بدخیم و خوشخیم، نزدیک به 1414 تصویر کوچک تر بدست آمد تا در نهایت مورد استفاده قرار گیرد.




%\section{پیش‌پردازش داده‌ها}\label{sec:پیش-پردازش-داده-ها}
\subsection{استخراج قسمت های کوچیک تر از یک اسلاید}\label{subsec:استخراج-قسمت-های-کوچیک-تر-از-یک-اسلاید}

اسلاید هایی که متخصصان از آن برای تشخیص استفاده می کنند بسیار ابعاد بزرگی دارند، به طوری که به هیچ وجه در
حافظه پردازنده ها جای نمی گیرد.
برای حل این مشکل در این پروژه سعی بر این شد که اسلاید‌ها به تیکه‌های کوچک تری از عکس شکسته شوند به طوری که قابل پردازش باشند.
یکی از موضوعات مهمی که باید در نظر گرفت، ناحیه‌های غیر مهم و فاقد اطلاعات است که باید در این فرآیند حذف گردند. این کار به دو دلیل انجام می شود.
اولا اینکه این قسمت‌ها، بخش بزرگی از اسلاید‌ها را تشکیل می دهند و با حذف این موارد می توان در استفاده از منابع پردازشی صرفه جویی کرد
و دوما اگر تعداد زیادی از عکس‌هایی که هنگام آموزش مدل استفاده می کنیم، از این نوع باشند، دقت مدل بشدت کاهش پیدا می کند و مدل در فرآیند یادگیری با مشکل مواجه می شود.
دلیل این امر هم این است که این عکس‌ها حاوی ویژگی‌های مورد نظر ما برای تشخیص سلول‌های سرطانی نیستند و مدل در حین فرآیند آموزش، ویژگی‌های نامربوطی را از روی این عکس‌ها یاد می گیرد.

برای حل این موضوع، روشی که بکار گرفته شد، استفاده از واریانیس لاپلاسین ناحیه است.
لاپلاسین یک عکس از محاسبه مشتق دوم روی شدت رنگ‌های پیکسل‌های آن محاسبه می شود
و در نتیجه لبه و گوشه‌های عکس مقدار بیشتری می گیرد.
برای محاسبه مشتق دوم برای یک عکس از رابطه زیر استفاده می کنیم که در آن $f(x)$ مقدار شدت رنگ را در موقعیت x از تصویر نشان می دهد.

\begin{gather*}
    f'(x) = f(x+1) - f(x), f'(x+1) = f(x+2) - f(x+1)\\
    f"(x) = f'(x+1) - f'(x)= f(x+2) - f(x+1) - f(x+1) + f(x)\\
    f"(x) = f(x+2) - 2*f(x+1) + f(x)\\
\end{gather*}

بعد از محاسبه واریانس شدت رنگ پیکسل‌ها در لاپلاسین عکس و استفاده از یک آستانه عکس‌هایی که مقدار کمتری دارند فیلتر می شوند.
به این ترتیب ناحیه‌های فاقد اطلاعات که مقدار لاپلاسین کمی نیز دارند حذف می شوند.

برای بدست آوردن مقدار بهینه آستانه و همچنین آزمودن این روش، به ترتیب زیر عمل شد.
\begin{enumerate}
    \item ابتدا حدس اولیه 500 برای آستانه انتخاب شد و اسلاید‌ها با توجه به این آستانه به قطعه‌های کوچک عکس در آمدند.
    مقدا اولیه آستانه در این مرحله، نیاز به دقت بالایی ندارد، زیرا همانطور که در قسمت بعد نیز توضیح داده خواهد شد، از آن برای انتخاب اسلاید‌ها استفاده می کنیم.
    این حدس اولیه نیز با مشاهده مقدار به صورت حدودی و از روی چند ناحیه تصادفی از یک اسلاید انتخاب شد.
    توزیع تعداد و درصد قطعه عکس‌های هر اسلاید در تصویر~\رجوع{شکل: توزیع تعداد و درصد قطعه عکس‌های هر اسلاید با آستانه 500} آمده است.
    \begin{figure}
        \begin{center}
            \includegraphics[width=0.48\linewidth]{figs/introduction/subs/challenges/patch_distribution_old_500_threshold.jpeg}
            \includegraphics[width=0.48\linewidth]{figs/introduction/subs/challenges/patch_percent_distribution_old_threshold_500.jpeg}
            \hspace{.2cm}
            \includegraphics[width=0.48\linewidth]{figs/introduction/subs/challenges/patch_distribution_new_298_threshold.jpeg}
            \includegraphics[width=0.48\linewidth]{figs/introduction/subs/challenges/patch_percent_distribution_new_threshold_298.jpeg}
        \end{center}
        \caption{نمودار‌ها به ترتیب از راست به چپ توزیع تعداد قطعه عکس‌های تولید شده از هر اسلاید و درصد قطعه عکس‌های تولید شده نسبت به تعداد کل را برای هر اسلاید با آستانه 500 و 298 نشان می دهد.}
        \label{شکل: توزیع تعداد و درصد قطعه عکس‌های هر اسلاید با آستانه 500}
    \end{figure}
    \item سپس سه اسلاید از پایین بازه و سه اسلاید از بالای بازه مشخص شده در نمودار توزیع قطعه عکس‌ها انتخاب شدند.
    سه اسلاید بالای بازه اسلاید‌هایی هستند که روش بکار رفته تعداد زیادی قطعه عکس از آن‌ها تولید کرده است و سه اسلاید پایین بازه اسلاید‌هایی هستند که به دلیل نابهینه بودن آستانه، روش ما، قطعه عکس‌های کمی را برای آن‌ها تولید کرده است.
    \item برای شش اسلاید انتخاب شده، به صورت دستی و با استفاده از نرم افزار GIMP ماسک‌هایی تولید شد.
    \item در قدم بعد، باید معیار‌هایی را برای روش بکار رفته انتخاب کرد تا با استفاده از آن‌ها بتوان آستانه بهینه را پیدا کرد و در نهایت آن را ارزیابی کرد.
    همانطور که در قسمت‌های قبل گفته شد، فیلتر ناحیه‌های فاقد اطلاعات، اهمیت زیادی برای ما دارد از این رو، در ارزیابی این روش علاوه بر دقت\LTRfootnote{Accuracy}، صحت\LTRfootnote{Precision} نیز عامل مهمی در کارکرد درست است.
    صحت، با توجه به فرمول زیر هرچه به مقدار عددی 1 نزدیک تر باشد به این معناست که ناحیه‌های بدست آمده از این روش، به احتمال بالاتری دارای اطلاعات هستند و در نتیجه ناحیه‌های فاقد اطلاعات کمتری تولید می شوند.
    \begin{gather*}
        TruePositive(TP) :\textit{روش به درستی عکس را غیر پس زمینه تشخیص داده}\\
        TrueNegative(TN) :\textit{ روش به درستی عکس را پس زمینه تشخیص داده}\\
        FalsePositive(FP) :\textit{ روش اشتباهاً عکس را غیر پس زمینه تشخیص داده}\\
        FalseNegative(FN) :\textit{ روش اشتباهاً عکس را پس زمینه تشخیص داده}\\
        Precision = \frac{TP}{TP + FP}\\
    \end{gather*}
    دقت روش نیز به صورت زیر محاسبه می شود:
    \[Accuracy = \frac{TP + TN}{TP + FP + TN + FN}\]
    \item حال با توجه به این دو معیار، قطعه کد پایتونی نوشته شد که شش اسلاید و ماسک‌های مربوط به آن‌ها را به عنوان ورودی می گیرد و با شروع از آستانه 500 و محاسبه ماتریس درهم ریختگی\LTRfootnote{Confusion Matrix}، در جهتی آستانه را تغییر می دهد تا دو معیار گفته شده بیشینه شوند.
    لازم به ذکر است، در نهایت از تابع هدف\LTRfootnote{Objective Function} $0.25*Precision+0.75*Accuracy$ برای یافتن آستانه استفاده شد تا صحت مقدار پایینی به خود نگیرد.
    در هر مرحله از اجرای کد، از هر اسلاید 2000 و در مجموع $2000 * 6$ قطعه عکس مورد بررسی قرار می گیرد به صورتی که آستانه در جهت افزایش تابع هدف و با اندازه پرش\LTRfootnote{Jump Size} $120$ و نرخ نزولی\LTRfootnote{Decay Rate} $0.85$، کاهش و یا افزایش پیدا می کند.
    نمودار تغییرات تابع هدف و آستانه در طی اجرای برنامه در تصویر~\رجوع{شکل: تغیرات تابع هدف و آستانه در طول اجرا} آمده است که در آن‌ها تابع هدف و آستانه به ترتیب به مقادیر $0.96$ و $298$ همگرا شده اند.
    \begin{figure}
        \begin{center}
            \includegraphics[width=0.48\linewidth]{figs/introduction/subs/challenges/laplacian_threshold_score_history_chart.jpeg}
            \includegraphics[width=0.48\linewidth]{figs/introduction/subs/challenges/laplacian_threshold_history_chart.jpeg}
        \end{center}
        \caption{به ترتیب از راست به چپ، تغیرات تابع هدف و آستانه قابل مشاهده است که در نهایت آستانه به مقدار 298 همگرا شده است.}
        \label{شکل: تغیرات تابع هدف و آستانه در طول اجرا}
    \end{figure}
    اسلاید‌ها و ماسک‌های بکار رفته آن‌ها در تصویر~\رجوع{شکل: شش اسلاید و ماسک‌های مرتبط} ارائه شده است و دقت و حساسیت روش بر روی این شش اسلاید نیز در جدول~\رجوع{جدول: دقت روش لاپلاسین بر روی شش اسلاید} آمده است.
    \begin{table}[t]
        \centering
        \begin{latin}
            \begin{tabular}{|c|l|c|c|c|}
                \hline
                \rl{اسلاید} & \rl{{ماتریس درهم ریختگی}} & \rl{دقت} & \rl{صحت}
                \\
                \hline
                \hline
                \textit{1} & \textit{TP: $4624$  FP: $126$ TN: $14100$ FN: $226$} & $0.98$ & $0.97$\\
                \textit{2} & \textit{TP: $1103$  FP: $172$ TN: $3758$  FN: $7$} & $0.96$ & $0.86$\\
                \textit{3} & \textit{TP: $7615$  FP: $92$  TN: $20871$ FN: $234$} & $0.98$ & $0.98$\\
                \textit{4} & \textit{TP: $78$    FP: $18$  TN: $1880$  FN: $4$} & $0.98$ & $0.81$\\
                \textit{5} & \textit{TP: $1138$  FP: $4$   TN: $6671$  FN: $492$} & $0.94$ & $0.99$\\
                \textit{6} & \textit{TP: $460$   FP: $0$   TN: $19618$ FN: $1426$} & $0.93$ & $0.99$\\
                \hline
                \textit{In Total} & \textit{TP: $15018$ FP: $412$ TN: $66898$ FN: $2389$} & $0.96$ & $0.97$\\
                \hline
            \end{tabular}
        \end{latin}
        \caption{دقت روش لاپلاسین برای تشخیص عکس‌های پس زمینه از غیر پس زمینه}
        \label{جدول: دقت روش لاپلاسین بر روی شش اسلاید}
    \end{table}

    \begin{figure}
        \begin{center}
            \includegraphics[width=0.48\linewidth]{figs/introduction/subs/challenges/evaluate_slides/TCGA-DJ-A1QG-01A-01-TSA.04c62c21-dd45-49ea-a74f-53822defe097__2000_generated_mask.jpg}
            \includegraphics[width=0.48\linewidth]{figs/introduction/subs/challenges/evaluate_slides/TCGA-DJ-A1QG-01A-01-TSA.04c62c21-dd45-49ea-a74f-53822defe097__2000_masked.png}
            \hspace{.2cm}
            \includegraphics[width=0.48\linewidth]{figs/introduction/subs/challenges/evaluate_slides/TCGA-EL-A3TB-11A-01-TS1.6E0966C9-1552-4B30-9008-8ACF737CA8C3__2000_generated_mask.jpg}
            \includegraphics[width=0.48\linewidth]{figs/introduction/subs/challenges/evaluate_slides/TCGA-EL-A3TB-11A-01-TS1.6E0966C9-1552-4B30-9008-8ACF737CA8C3__2000_masked.png}
            \hspace{.2cm}
            \includegraphics[width=0.48\linewidth]{figs/introduction/subs/challenges/evaluate_slides/TCGA-ET-A39O-01A-01-TSA.3829C900-7597-4EA9-AFC7-AA238221CE69_7000_generated_mask.jpg}
            \includegraphics[width=0.48\linewidth]{figs/introduction/subs/challenges/evaluate_slides/TCGA-ET-A39O-01A-01-TSA.3829C900-7597-4EA9-AFC7-AA238221CE69_7000_masked.png}
            \hspace{.2cm}
            \includegraphics[width=0.48\linewidth]{figs/introduction/subs/challenges/evaluate_slides/TCGA-EL-A4K7-11A-01-TS1.C08B59AA-87DF-4ABB-8B70-25FEF9893C7F__70_generated_mask.jpg}
            \includegraphics[width=0.48\linewidth]{figs/introduction/subs/challenges/evaluate_slides/TCGA-EL-A4K7-11A-01-TS1.C08B59AA-87DF-4ABB-8B70-25FEF9893C7F__70_masked.png}
            \hspace{.2cm}
            \includegraphics[width=0.48\linewidth]{figs/introduction/subs/challenges/evaluate_slides/TCGA-ET-A39N-01A-01-TSA.C38FCE19-9558-4035-9F0B-AD05B9BE321D___198_generated_mask.jpg}
            \includegraphics[width=0.48\linewidth]{figs/introduction/subs/challenges/evaluate_slides/TCGA-ET-A39N-01A-01-TSA.C38FCE19-9558-4035-9F0B-AD05B9BE321D___198_masked.png}
            \hspace{.2cm}
            \includegraphics[width=0.48\linewidth]{figs/introduction/subs/challenges/evaluate_slides/TCGA-BJ-A3F0-01A-01-TSA.728CE583-95BE-462B-AFDF-FC0B228DF3DE__3_generated_mask.jpg}
            \includegraphics[width=0.48\linewidth]{figs/introduction/subs/challenges/evaluate_slides/TCGA-BJ-A3F0-01A-01-TSA.728CE583-95BE-462B-AFDF-FC0B228DF3DE__3_masked.png}
        \end{center}
        \caption{ناحیه‌های استخراج شده شش اسلاید $1$, $2$, $3$, $4$, $5$, $6$ در آستانه 298 و ماسک‌های مرتبط با آن‌ها}
        \label{شکل: شش اسلاید و ماسک‌های مرتبط}
    \end{figure}
\end{enumerate}
بعد از یافتن آستانه، می توان اسلاید‌ها را یکی پس از دیگری خواند و نواحی مورد نیاز برای آموزش مدل را استخراج کرد.
با توجه به توضیحات داده شده برای این روش استخراج، در ادامه به دو تا از مزیت‌های آن اشاره می کنیم:
\begin{itemize}
    \item از آنجایی که حجم داده ای که با آن‌ها کار داریم بسیار زیاد است و اسلاید‌های زیادی نیاز به پیش پردازش دارند سرعت و دقت عامل مهمی در عملکرد است.
    در این روش، بعد از تهیه ماسک‌ها برای شش اسلاید ذکر شده به صورت دستی و تعیین آستانه، دیگر نیاز به دخالت انسان نداریم، از این روی، این روش دقت و سرعت بالایی را داراست.
    \item همانطور که پیشتر نیز گفته شد، فایل‌های دیجیتال اسلاید‌ها، از طریق اسکن نمونه‌ها توسط دستگاه و در زیر میکروسکپ تهیه می شوند.
    در بعضی مواقع به دلیل شرایط فیزیکی نمونه، خطای دستگاه و یا حتی خطای انسانی ممکن است نمونه و یا قسمتی از آن به درستی اسکن نشود.
    این نواحی نیز علاوه بر نواحی پس زمینه اطلاعات مورد نیاز ما را ندارند و یا از دست داده اند و حالت بلوری به خود گرفته اند.
    با توجه به ماهیت روش استخراج ذکر شده، این مشکل در این فرآیند حل می شود.
    لاپلاسین یک تصویر حاوی اطلاعاتی نظیر گوشه‌ها و خطوط تصویر است از این روی، یکی از روش‌های اصلی تشخیص بلوری بودن تصویر است، زیرا در این تصاویر اطلاعاتی مانند خطوط محو می شوند.
    با توجه به اینکه مبنای اصلی روش استخراج ذکر شده نیز لاپلاسین تصویر است، در طی این فرآیند تصاویر بلوری مقدار واریانس لاپلاسین کمتری گرفته و خود به خود حذف می گردند.
\end{itemize}

\section{داده افزایی}\label{subsec:داده-افزایی}

داده افزایی یکی از پیش‌پردازش‌هایی ست که قبل از داده شدن داده‌ها به مدل، روی آن‌ها انجام می‌شود.
در داده افزایی\LTRfootnote{Data Augmentation} از تصاویر موجود، تصاویر جدیدی بازآفرینی می‌شوند.
معمولا تصاویر جدید ایجاد شده، توزیع داده نزدیکی به توزیع داده ی داده‌های اصلی و تست دارند و هدف از این کار این است که تعمیم یافتگی مدل را افزایش دهیم تا دقت مدل بر روی داده‌های تست افزایش پیدا کند.
روش‌ها و الگوریتم‌های مختلف با پیچیدگی‌های مختلفی وجود دارد که در ادامه روش‌هایی که در این پروژه مورد بررسی قرار گرفت، آمده است.

\textbf{چرخش تصویر به صورت تصادفی}:
در این روش تصاویر به اندازه مقداری تصادفی چرخ داده می‌شوند\LTRfootnote{Random Rotate}.
از آنجایی که در این مسئله، با چرخش، تصاویر اطلاعاتی از دست نمی دهند و علاوه بر آن، مقدار خروجی مدل ما نیز نباید تغییر کند، لذا می‌توان از این روش برای داده افزایی استفاده کرد.

\textbf{آینه کردن به صورت تصادفی}:
مانند قسمت قبل،از این روش نیز می‌توان برای داده افزایی استفاده کرد. تصاویر در این روش چپ به راست یا بالا به پایین معکوس می‌شوند طوری که انگار جلوی آیینه قرار گرفته‌اند.\LTRfootnote{Random Flip}.

\textbf{تغییر مقیاس به صورت تصادفی}:
معمولا عکس‌هایی که از آن‌ها برای آموزش مدل استفاده می‌کنیم دارای ابعاد و رزولوشن متفاوتی هستند و در نتیجه اندازه سلول‌ها در آن‌ها متفاوت است.
از آنجایی که مدل در نهات تلاش می‌کند ویژگی‌های مرتبط با این سلول‌ها را بیابد و با توجه به آن‌ها، تشخیص درستی را ارایه دهد، باید پیش‌پردازشی انجام گیرد تا مدل نسبت به اندازه سلول‌ها، حساس نشود. از این روی می‌توان قبل از استفاده از عکس‌ها در آموزش مدل، مقیاس تصاویر را به صورت تصادفی تغییر داد\LTRfootnote{Random Scale}.

\textbf{بلورتر کردن به صورت تصادفی و اضافه کردن نوفه گوسین }:
ممکن است تصاویر اسکن شده، به خوبی تصاویر زمان آموزش مدل نباشد و عوامل محیطی، باعث به وجود آمدن نوفه\LTRfootnote{Noise} در داده‌ها شده باشد، از این روی ما از قصد نویز و یا بلور رو به عکس زمان آموزش اضافه می‌کنیم تا مدل توانایی بیشتری برای پیشبینی درست روی عکس‌های نویزی داشته باشد.

\textbf{تغییر رنگ}:
همانطور که پیشتر نیز گفته شد، عکس‌ها و اسلاید‌ها ممکن است با روش‌های متفاوتی رنگ شده باشند و رنگ‌های مختلفی را به خود بگیرند.
از این روی باید عکس‌های زمان آموزش مدل نیز به اندازه کافی تنوع رنگ داشته باشد و مدل توانایی تشخیص درست در بازه رنگ‌های متفاوتی را داشته باشد.
برای این کار می‌توان رنگ‌آمیزی عکس‌ها را پیش از دادن به مدل تغییر داد.
روشی که در این جا استفاده می‌شود به این صورت است که در ابتدا عکس‌ها را از حالت \lr{RGB} به \lr{HSV} تبدیل می‌کنیم.
در دامنه \lr{HSV}، کانال‌ها به ترتیب حاوی اطلاعات رنگ\LTRfootnote{Hue}، اشباع\LTRfootnote{Saturation} و روشنایی\LTRfootnote{Brightness} هستند.
در اینجا کافیست مقدار کانال رنگ را که مقداری بین 0 تا 360 به خود می‌گیرد را صورت تصادفی تغییر دهیم.
در تصویر \ref{jitter augmentation} مثالی از این پیش‌پردازش آمده است.
\begin{figure}
    \begin{center}
        \includegraphics[width=0.48\linewidth]{figs/suggested_methods/subs/data_augmentation/jitter_1054-original.jpeg}
        \includegraphics[width=0.48\linewidth]{figs/suggested_methods/subs/data_augmentation/jitter_1054-transformed.jpeg}
    \end{center}
    \caption[نمونه‌ای از داده‌افزایی تغییر رنگ]{ به ترتیب از راست به چپ عکس اصلی و عکس تبدیل‌شده آمده است.}
    \label{jitter augmentation}
\end{figure}

\subsubsection{تطبیق دامنه فوریه}
%https://towardsdatascience.com/deep-domain-adaptation-in-computer-vision-8da398d3167f
در هنگام آموزش یک مدل، فرض می‌شود که داده‌های آموزشی (چه بزرگ یا کوچک) نماینده خوبی از توزیع کلی داده‌هاست.
با این حال، اگر ورودی‌ها در زمان آزمون به طور قابل توجهی با داده‌های آموزشی متفاوت باشد، مدل ممکن است عملکرد چندان خوبی نداشته باشد در حالی که برای یک انسان، با یادگیری مفهوم یک موضوع این مشکل کمتر وجود دارد.
دلیل اینکه مدل شما در این سناریوها خیلی خوب عمل نمی کند این است که دامنه مسئله تغییر کرده است. در این مورد تطبیق دامنه به کمک شما می‌آید. تطبیق دامنه زیرشاخه‌ای از یادگیری ماشین است که به سناریوهایی می‌پردازد که در آن یک مدل آموزش‌دیده بر روی توزیع منبع در زمینه توزیع هدف متفاوت استفاده می‌شود. به طور کلی، تطبیق دامنه\LTRfootnote{Domain adaptation} از داده‌های برچسب گذاری‌شده در یک یا چند دامنه منبع برای حل وظایف جدید در یک دامنه هدف استفاده می‌کند.
\newline
به طور مثال در مسئله این پروژه تنظیمات دستگاه اسکنر و نوع آن، روش اسکن، روش نمونه برداری و ... ممکن است روی توزیع داده‌ها تاثیر بگذارد در حالی که این عوامل در بیمارستان‌های مختلف و مجموع‌داده‌های مختلف متفاوت است، لذا استفاده از تطبیق دامنه، گزینه خوبی برای افزایش عملکرد مدل است.

یکی از روش‌های تطبیق دامنه، تطبیق دامنه فوریه\LTRfootnote{Fourier domain adaptation} است. این روش ابتدا در سال 2020 میلادی و در مقاله \cite{yang2020fda} معرفی شد و به این صورت عمل می‌کند که ابتدا عکس‌های هدف و منبع را با تبدیل فوریه به حوزه فرکانس می‌برد، سپس ناحیه با فرکانس پایین در داده‌های منبع را با ناحیه‌های هدف جاگزین می‌کند و بعد از آن فوریه معکوس را روی داده‌ها انجام داده و داده‌های به حالت قبل بر می‌گردند. دلیل و انگیزه این روش است که طیف دامنه سطح پایین حوزه فرکانس می‌تواند به طور قابل توجهی تغییر کند بدون اینکه بر درک معناشناسی سطح بالا تأثیر بگذارد و با این تغییر، توزیع تصویر نهایی به تصویر هدف نزدیک می‌شود.
این روش بااینکه هزینه محاسباتی بسیار کمی دارد، اما در مواردی عملکرد خوبی را ارائه کرده است که نمونه از عملکرد این روش، در تصویر \ref{fda augmentation} آمده است.
\begin{figure}
    \begin{center}
        \includegraphics[width=0.48\linewidth]{figs/suggested_methods/subs/data_augmentation/fda_1054-original.jpeg}
        \includegraphics[width=0.48\linewidth]{figs/suggested_methods/subs/data_augmentation/fda_1054-transformed.jpeg}
    \end{center}
    \caption[نمونه‌ای از داده‌افزایی تطبیق دامنه فوریه]{ به ترتیب از راست به چپ عکس اصلی و عکس تبدیل‌شده آمده است.}
    \label{fda augmentation}
\end{figure}

\subsubsection{ادغام تصاویر}
ادغام تصویر\LTRfootnote{Mix-up} یکی از روش‌های افزایش داده است که ابتدا در سال 2017 مقاله
\cite{zhang2017mixup}
معرفی شد.
این الگوریتم کاملاً نظام‌مند نامگذاری شده است که در آن به معنای واقعی کلمه ویژگی‌ها و برچسب‌های مربوط به آنها را با هم مخلوط می‌کنیم. شبکه‌های عصبی، مستعد به خاطر سپردن برچسب‌های اشتباه هستند. رویه ذکر‌شده این کار را با ترکیب ویژگی‌های مختلف با یکدیگر کاهش می‌دهد (همین مورد برای برچسب‌ها نیز اتفاق می‌افتد) به طوری که یک شبکه در مورد رابطه بین ویژگی‌ها و برچسب‌های آنها بیش از حد مطمئن نشود.
همانطور که گفته شد در ادغام دو عکس را با ضرایب $\lambda$ و ۱-$\lambda$ بین صفر و یک، پیکسل به پیکسل با هم جمع می‌کنند تا عکس جدیدی بدست آید. مقدار $\lambda$ طوری انتخاب می‌شود که تصویر اصلی دچار تغییر معنا نشود.
نمونه‌ای از نتایج این روش در تصویر \ref{mixup augmentation} آمده است.
\begin{figure}
    \begin{center}
        \includegraphics[width=0.48\linewidth]{figs/suggested_methods/subs/data_augmentation/mixup_776-original.jpeg}
        \includegraphics[width=0.48\linewidth]{figs/suggested_methods/subs/data_augmentation/mixup_776-transformed.jpeg}
    \end{center}
    \caption[نمونه‌ای از داده‌افزایی ادغام تصویر]{ به ترتیب از راست به چپ عکس اصلی و عکس تبدیل‌شده آمده است.}
    \label{mixup augmentation}
\end{figure}


\section{آموزش مدل}\label{sec:آموزش مدل}
\subsection{مدل‌های مورد بررسی}\label{subsec:مدل‌های مورد بررسی}
در این پژوهش، از دو نوع مدل با ایده اصلی متفاوت و ظرفیت شبکه‌های متفاوت استفاده شد، تا بتوان نتیجه نهایی را برای هر کدام مقایسه کرد.
این مدل‌ها \lr{InceptionV3}\cite{szegedy2016rethinking}، \lr{InceptionV4}\cite{szegedy2017inception}، \lr{ResNet18}\cite{he2016deep} و \lr{ResNet101}\cite{he2016deep} هستند. 
در این قسمت به معرفی این دو مدل می‌پردازیم.

\subsubsection{معماری \lr{Inception}}
این معماری ابتدا در \cite{szegedy2015going} معرفی شد که یک شبکه عمیق مبتنی بر شبکه‌های عصبی پیچشی است. این شبکه از اجزای تکرارشونده با نام ماژول \lr{Inception} تشکیل شده است. هر کدام از این ماژول‌ها، ورودی را با ماتریس‌های هسته با اندازه متفاوت پردازش می‌کنند. به این معنی که بجای اینکه در هر لایه از شبکه \lr{CNN} از هسته با یک سایز استفاده شود در هر ماژول از هسته‌هایی با سایزهای متفاوت استفاده می‌شود و در نهایت خروجی‌ها را با هم ادغام می‌کنیم. یکی از اهداف این کار پیدا کردن فیچرها در مقیاس‌های متفاوت است. جزییات بیشتر این معماری در مقاله اشاره‌شده آمده است.


\subsubsection{معماری \lr{ResNet}}
مبنای این معماری نیز شبکه‌های عصبی پیچشی است که ابتدا در پژوهش \cite{he2016deep} معرفی شد.
یکی از مشکلاتی که شبکه‌های عصبی به خصوص پیچشی با آن روبرو است، نیاز به استفاده از تعداد لایه‌های زیاد برای یادگیری ویژگی‌های مناسب مسئله است.
با افزایش تعداد لایه‌ها در این شبکه‌ها، قابلیت شبکه برای پیدا کردن ویژگی‌های پیچیده بیشتر می‌شود اما افزایش عمق شبکه، تنها با انباشتن لایه‌ها در کنار هم کار نمی‌کند.
آموزش شبکه‌های عمیق این چنینی به دلیل مشکل ناپدید شدن گرادیان بسیار دشوار است زیرا، در حین فرآیند آموزش که گرادیان به لایه‌های قبلی انتشار می‌یابد، ضرب مکرر ممکن است گرادیان را بینهایت کوچک کند و در چنین حالتی لایه‌های اول به درستی آموزش نمی‌بینند.
این معماری با رها کردن تصادفی و پرش از لایه‌های آن در حین آموزش و استفاده از شبکه کامل در زمان آزمایش، روشی ضد شهودی را برای آموزش یک شبکه بسیار عمیق پیشنهاد کرد. برای این کار در بعضی نقاط از شبکه، ارتباطات مستقیم بین دو لایه غیر مجاور وجود دارد. این ارتباطات، خروجی لایه قبل را به ورودی لایه پایان ارتباط اضافه می‌کند. با این کار از همگرا شدن گرادیانت به صفر جلوگیری می‌کند. جزییات بیشتر این معماری در مقاله اصلی آمده است.
\subsection{روش آموزش مدل‌ها}\label{subsec:روش آموزش مدل‌ها}
برای فرآیند آموزش، مدل‌های معرفی شده در بخش قبل هرکدام به صورت جداگانه مورد و مستقل آموزش داده شدند. این مدل‌ها بر روی دو مجموع‌داده پاپسوسایتی و پایگاه داده ریزآرایه استنفورد و روش‌های مختلف داده افزایی مورد آزمایش قرار گرفتند. که در مجموع 1258 تصویر برای آموزش، 174 تصویر برای ارزیابی و 378 تصویر نیز برای تست به‌کار گرفته شد. نکته‌ای که در هنگام تقسیم بندی تصاویر به سه گروه آموزش، ارزیابی و تست باید در نظر گرفت، این است که نباید تصویری از یک اسلاید در دو گروه قرار بگیرند. به عنوان مثال، اگر از یک اسلاید چند تصویر استخراج شود باید تصاویر استفاده‌شده تنها در یکی از این سه گروه قرار بگیرند زیرا این تصاویر توزیع بسیار نزدیکی بهم دارند در نتیجه، قرار دادن آن‌ها در یک گروه به ارزیابی بهتر مدل کمک می‌کند.
نتایج نهایی اجراها در جدول \ref{table:papsociety_and_stanford_run_results} آمده است.
در ستون اول این جدول مدل‌های مورد آزمایش، در ستون دوم روش‌های داده افزایی مختلف، ستون سوم بهترین دور در حین فرآیند آموزش، ستون چهارم دقت نهایی در تشخیص بدخیم و یا خوشخیم بودن تصویر و ستون آخر نیز ویژگی عملکرد گیرنده\LTRfootnote{Receiver operating characteristic - ROC} برای گروه بدخیم آمده است. دقت به صورت میانگین دقت مدل در دو گروه محاسبه شده است.
روش‌های داده افزایی که در ستون دوم ذکر شده و به‌کار گرفته شد به چند گروه متفاوت دسته بندی می‌شوند. منظور از \lr{mixup} همان روش ادغام تصاویر است. منظور از \lr{fda} نیز روش تطبیق دامنه فوریه است و \lr{jit} نیز تغییر تصادفی رنگ تصویر است و در نهایت منظور از \lr{all} ترکیبی از هر سه روش است. در این ستون دو کلمه دیگر نیز دیده می‌شود. کمله \lr{base} اشاره به ترکیبی از روش‌های آیینه کردن، چرخاندن، تغییر مقیاس و اضافه کردن نویز گوسی دارد و کلمه \lr{base-nrs} تنها تفاوتی که در روش داده‌افزایی دارد عدم وجود روش تغییر مقیاس است. در نهایت \lr{none} اشاره به عدم استفاده از داده‌افزایی می‌کند.

در ادامه از مجموع‌داده اطلس ژنوم سرطان تیروئید برای آموزش مدل \lr{ResNet101} استفاده شد. همانطور که در فصل‌های پیش نیز اشاره شد، برچسب‌های این مجموع‌داده به صورت در صد سلول‌ها نرمال و توموری تهیه شده است. در ادامه سعی شده تا دسته بندی بر روی این مجموع‌داده آموزش داده شود تا با داشتن تصویر، وجود یا عدم وجود سلول‌های توموری را تشخیص دهد. تعداد تصاویری که در نهایت از این مجموع‌داده استخراج شد، 265418 است که از 185090 تصویر برای آموزش، 25909 تصویر برای ارزیابی و 54419 برای تست استفاده شد. باز باید روی این نکته تاکید کرد که تصاویر استخراج‌شده از یک اسلاید نباید در دو گروه قرار بگیرند.
با توجه به حجم زیاد داده‌ها و زمان بر بودن پردازش، در این قسمت تنها از روش \lr{mixup} و \lr{base} برای پیش‌پردازش استفاده شد. مدل در $20$ دور\LTRfootnote{Epoch} بر روی داده‌ها آموزش داده شد. دقت و نمودار‌های مربوط به آن در تصویر \ref{nci_dataset_with_resnet101_results} آمده است. دستگاه گرافیک به‌کار رفته مدل \lr{Quadro RTX 8000} بوده و زمان اجرا $32$ ساعت طول کشیده است.





