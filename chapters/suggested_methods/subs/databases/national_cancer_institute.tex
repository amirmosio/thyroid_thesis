\subsection{اطلس ژنوم سرطان تیروئید}\label{subsec:موسسه-ملی-سرطان-پورتال-داده-های-مشترک-ژنومیک}

اطلس ژنوم سرطان تیروئید\LTRfootnote{The Cancer Genome Atlas-Thyroid Cancer/TCGA-THCA} بخشی از یک تلاش بزرگتر برای ایجاد یک جامعه تحقیقاتی متمرکز بر اتصال فنوتیپ‌های\LTRfootnote{Phenotypes} سرطان به ژنوتیپ‌ها\LTRfootnote{Genotypes} با ارائه تصاویر بالینی منطبق با افراد از اطلس ژنوم سرطان\cite{clark2013cancer} (TCGA) است.
%\lr{Cite: "The results <published or shown> here are in whole or part based upon data generated by the TCGA Research Network: %https://www.cancer.gov/tcga."}
این پروژه شامل اطلاعات 507 بیمار است که هدف اصلی آن، بررسی سرطان تیروئید در این بیماران است.
این اطلاعات شامل توالی دی ان ای، جنسیت، سن، داده‌های بالینی از بررسی بیمار توسط پزشک، اسلایدهای دیجیتال نمونه برداری‌شده از غده، درصد سلول‌های توموری و سالم و بسیاری موارد دیگر است.
تعداد کل اسلاید‌های جمع آوری‌شده مورد استفاده ما از این پروژه 639 عدد است که در مجموع حجمی بالغ بر 105 گیگابایت دارند. برای هرکدام از این اسلاید‌ها سه برچسب درصد سلول‌های عادی، درصد سلول‌های سرطانی و درصدر سلول‌ها استورمال آمده است که ما نیز از این برچسب‌ها برای مسئله خود استفاده می‌کنیم. از بین این اسلاید‌ها و با توجه به درصد سلول‌ها نرمال، توموری و استورمال، تعداد کل 182 اسلاید انتخاب شد که در بین آن‌ها، 106 اسلاید صددرصد توموری و 76 اسلاید آن، صددرصد عادی است.
%در مرحله بعد برای استفاده از این 182 اسلاید، آن‌ها، به 265418 قطعه عکس کوچک تر در آورده شدند تا بعد از آن بتوانند در آموزش مدل مورد %استفاده قرار بگیرند.
نمونه‌ای از اسلاید این مجموع‌داده در \ref{fig:sampleWSIscan}  و در فصل قبل آمده است.
در صورت نیاز، جزئیات بیشتری از این مجموع‌داده در آدرس \cite{ncigdc} قرار دارد.