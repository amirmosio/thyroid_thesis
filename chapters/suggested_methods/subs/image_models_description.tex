\subsection{مدل‌های مورد بررسی}\label{subsec:مدل‌های مورد بررسی}
در این پژوهش، از دو نوع مدل با ایده اصلی متفاوت و ظرفیت شبکه‌های متفاوت استفاده شد، تا بتوان نتیجه نهایی را برای هر کدام مقایسه کرد.
این مدل‌ها \lr{InceptionV3}\cite{szegedy2016rethinking}، \lr{InceptionV4}\cite{szegedy2017inception}، \lr{ResNet18}\cite{he2016deep} و \lr{ResNet101}\cite{he2016deep} هستند. 
در این قسمت به معرفی این دو مدل می‌پردازیم.

\subsubsection{معماری \lr{Inception}}
این معماری ابتدا در \cite{szegedy2015going} معرفی شد که یک شبکه عمیق مبتنی بر شبکه‌های عصبی پیچشی است. این شبکه از اجزای تکرارشونده با نام ماژول \lr{Inception} تشکیل شده است. هر کدام از این ماژول‌ها، ورودی را با ماتریس‌های هسته با اندازه متفاوت پردازش می‌کنند. به این معنی که بجای اینکه در هر لایه از شبکه \lr{CNN} از هسته با یک سایز استفاده شود در هر ماژول از هسته‌هایی با سایزهای متفاوت استفاده می‌شود و در نهایت خروجی‌ها را با هم ادغام می‌کنیم. یکی از اهداف این کار پیدا کردن فیچرها در مقیاس‌های متفاوت است. جزییات بیشتر این معماری در مقاله اشاره‌شده آمده است.


\subsubsection{معماری \lr{ResNet}}
مبنای این معماری نیز شبکه‌های عصبی پیچشی است که ابتدا در پژوهش \cite{he2016deep} معرفی شد.
یکی از مشکلاتی که شبکه‌های عصبی به خصوص پیچشی با آن روبرو است، نیاز به استفاده از تعداد لایه‌های زیاد برای یادگیری ویژگی‌های مناسب مسئله است.
با افزایش تعداد لایه‌ها در این شبکه‌ها، قابلیت شبکه برای پیدا کردن ویژگی‌های پیچیده بیشتر می‌شود اما افزایش عمق شبکه، تنها با انباشتن لایه‌ها در کنار هم کار نمی‌کند.
آموزش شبکه‌های عمیق این چنینی به دلیل مشکل ناپدید شدن گرادیان بسیار دشوار است زیرا، در حین فرآیند آموزش که گرادیان به لایه‌های قبلی انتشار می‌یابد، ضرب مکرر ممکن است گرادیان را بینهایت کوچک کند و در چنین حالتی لایه‌های اول به درستی آموزش نمی‌بینند.
این معماری با رها کردن تصادفی و پرش از لایه‌های آن در حین آموزش و استفاده از شبکه کامل در زمان آزمایش، روشی ضد شهودی را برای آموزش یک شبکه بسیار عمیق پیشنهاد کرد. برای این کار در بعضی نقاط از شبکه، ارتباطات مستقیم بین دو لایه غیر مجاور وجود دارد. این ارتباطات، خروجی لایه قبل را به ورودی لایه پایان ارتباط اضافه می‌کند. با این کار از همگرا شدن گرادیانت به صفر جلوگیری می‌کند. جزییات بیشتر این معماری در مقاله اصلی آمده است.