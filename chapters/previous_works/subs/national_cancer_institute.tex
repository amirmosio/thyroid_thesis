\subsection{اطلس ژنوم سرطان تیروئید}\label{subsec:موسسه-ملی-سرطان-پورتال-داده-های-مشترک-ژنومیک}

اطلس ژنوم سرطان تیروئید\LTRfootnote{The Cancer Genome Atlas-Thyroid Cancer/TCGA-THCA} بخشی از یک تلاش بزرگتر برای ایجاد یک جامعه تحقیقاتی متمرکز بر اتصال فنوتیپ‌های\LTRfootnote{Phenotypes} سرطان به ژنوتیپ‌ها\LTRfootnote{Genotypes} با ارائه تصاویر بالینی منطبق با افراد از اطلس ژنوم سرطان (TCGA) است.\cite{clark2013cancer}

\lr{Cite: "The results <published or shown> here are in whole or part based upon data generated by the TCGA Research Network: https://www.cancer.gov/tcga."}
شناسه پروژه به صورت \lr{TCGA-THCA} است که در آدرس \cite{ncigdc} قابل مشاهده است. این پروژه شامل اطلاعات 507 بیماراست که هدف اصلی آن، بررسی سرطان تیروئید در این بیماران است.

این اطلاعات شامل توالی دی ان ای، جنسیت، سن، داده‌های کلینیکال از بررسی بیمار توسط پزشک، اسلایدهای دیجیتال نمونه برداری‌شده از غده تیروئید و تعدادی موارد دیگر است.
تعداد کل اسلاید‌های جمع آوری‌شده مورد استفاده ما از این پروژه 639 عدد است که در مجموع حجمی بالغ بر 105 گیگابایت دارند. برای هرکدام از این اسلاید‌ها سه برچسب درصد سلول‌های عادی، درصد سلول‌های سرطانی و درصدر سلول‌ها استورمال آمده است که ما نیز از این برچسب‌ها برای مسئله خود استفاده می‌کنیم. از بین این اسلاید‌ها و با توجه به درصد سلول‌ها نرمال، سرطانی و استورمال 106 اسلاید صددرصد سرطانی و 76 اسلاید صددرصد عادی برای ادامه کار انتخاب شد.
در مرحله بعد برای استفاده از این 182 اسلاید، آن‌ها، به 265418 قطعه عکس کوچک تر در آورده شدند تا بعد از آن بتوانند در آموزش مدل مورد استفاده قرار بگیرند.