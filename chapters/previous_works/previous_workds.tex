\فصل{کارهای پیشین}
در سال‌های گذشته برای تشخیص سرطان تیروئید و گروه بندی آن‌های کارهای بسیاری مبتنی برای یادگیری ماشین انجام گرفته است. این پژوهش‌ها بر روی دیتاست‌های مختلف و با اهداف متفاوتی شکل گرفته اند. در این فصل به بررسی کارهای پیشین در زمینه موضوعات مرتبط می پردازیم.

به عنوان مثال در سال ۲۰۱۹ پژوهش \cite{li2019diagnosis} سعی بر این داشته تا با استفاد از تصاویر سونوگرافیک و مبتنی بر روش‌های شبک‌های عصبی پیچشی عمیق\LTRfootnote{Deep Convolutional Neural Network - DCNN}، سرطان تیروئید را تشخیص دهد.

در همان سال در پژوهش آزمایشی \cite{guan2019deep} گوان و همکاران تلاش کردندن تا با استفاده از اسلاید‌های دیجیتال سایتولوژی، مدل‌های \lr{VGG16} و \lr{InceptionV3} را آموزش دهند. هدف این پژوهش در ابتدا طبقه‌بندی سرطان تیروئید به سه گروه فولیکولار ادنوما\LTRfootnote{Follicular Adenoma} و فولیکولار کارسینوما\LTRfootnote{Follicular Carcinoma} و پاپیلاری کارسینوما\LTRfootnote{Papillary Carcinoma} بوده است که در مرحله بعد با همین معماری سعی کردندن دسته فولیکولار کارسینوما را به چهار زیر گروه دیگر طبقه‌بندی کنند. در نهایت بهترین نتیجه را با مدل \lr{VGG16} و با دقت 97.66 برروی تصاویر استخراج شده کسب کردند.

در سال ۲۰۲۱ نیز در پژوهش \cite{bohland2021machine} سعی بر این شد تا روش‌های مختلف و مدل‌های مختلف مورد بررسی قرار گیرد تا بتوان مقایسه را بین آن‌ها انجام داد. هدف اصلی این پژوهش تشخیص سرطان تیروئید از نوع پاپیلاری بوده است. دو روشی که در این پژوهش مورد بررسی قرار گرفتند به صورت زیر هستند.

\begin{itemize}
    \item طبقه بندی بر اساس ویژگی\newline
    قدم اول این روش، مرزبندی\LTRfootnote{Segmentation} است که در آن، تصاویر تیروئید برای استخراج بخش‌های تیروئید، قطعات هسته و ماسک‌های هسته استفاده می شوند. برای بدست آوردن مرزبندی مناسب از شبکه \lr{UNet} استفاده شد. این شبکه ابتدا در سال ۲۰۱۵ و در مقاله \cite{ronneberger2015u} معرفی شد که به دلیل داشتن ساختار U شکل در لایه‌های خود، شبکه ای بسیار مناسب برای مرزبندی تصاویر بیولوژیکی است.
    برای فرآیند استخراج ویژگی، تصاویر تیروئید، مرزبندی‌های مرتبط و ماسک‌های تهیه شده از سلول‌ها استفاده می شوند تا ویژگی‌های مختلف سلول بدست بیایند.
    در مرحله بعد، ویژگی‌های بدست آمده جمع می شوند و روی آن‌ها پیشپردازش انجام می گیرد تا مورد استفاده مدل برای طبقه بندی قرار گیرند. الگوریتم‌های متفاوت و کلاسیکی مانند طبقه‌بندی بردار پشتیبانی، نزدیک ترین همسایه، بیز ساده گوسی و ... برای این کار مورد آزمایش قرار گرفت که در نهایت بهترین آن‌ها انتخاب گردید.
    \item طبقه بندی مبتنی بر یادگیری عمیق\newline
    این روش، روش مستقیمی برای طبقه بندی به حساب می آید به همین دلیل نیاز به استخرج ویژگی به صورت دستی نیست و قدم‌ها مورد نیاز برای طبقه بندی به شدت کاهش پیدا می کنند. برای کاهش زمان آموزش مدل‌ها نیز، از مدل‌های از پیش آموزش داده شده برای روی دیتاست \lr{ImageNet}\cite{deng2009imagenet} استفاده شد. در نهایت قبل از استفاده از داده‌ها، روش‌های مختلف از جمله چرخش تصویر، آیینه کردن، متعادل سازی هسیتوگرارم تصویر\LTRfootnote{Contrast Limited Adaptive Histogram Equalization}، تطبیق دامنه فوریه و ... برای داده‌افزایی بکار گرفته شد.

\end{itemize}
نتایج نهایی این پژوهش به این صورت گزارش شد که بهترین نتیجه توسط روش مبتنی بر ویژگی با دقت 83 و 89 درصد بر روی دو دستاست، بدست آمد. در هر دو دیتاست دقت روش مبتنی بر ویژگی بالاتر از روش یادگیری عمیق بوده، اما در پایان، پیشنهاد شد که برای دیتاست‌های کوچک تر و مرزی تر(منظور از مرزی بودن داده در اینجا این است که به سادگی نمی توان به گروه‌های مورد نظر طبقه بندی کرد) از روش مبتنی بر ویژگی و برای دیتاست‌های بزرگ تر و کلی تر از روش مبتنی بر یادگیری همیق استفاده گردد.