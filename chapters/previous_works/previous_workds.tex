\فصل{کارهای پیشین}
در سال‌های گذشته برای تشخیص سرطان تیروئید و گروه‌بندی‌های آن‌ کارهایی مبتنی بر یادگیری ماشین انجام گرفته‌است.
این پژوهش‌ها بر روی مجموع‌داده‌های مختلف و با اهداف متفاوتی شکل گرفته‌اند.
در دسته‌ای از این ‍‍پژوهش‌ها هدف تشخیص کلی وجود یا عدم وجود سرطان تیروئید بوده‌است.
در گروهی دیگر هدف تشخیص نوع سرطان(پاپیلاری، فولیکولار، مدولاری) و زیر‌گروه‌های هر کدام بوده‌است.
گروهی تمرکز خود را بر روی نوعی خاص از انواع سرطان تیروئید گذاشته و وجود یا عدم وجود آن را در داده‌ها بررسی می کنند.
هدف تعدادی از این پژوهش‌ها پیشبینی معیار \lr{TBSR}\LTRfootnote{The Bethesda System for Reporting} بوده است. این معیار وضعیت غده تیروئید را به شش گروه 
در ادامه این فصل، به بررسی تعدادی از کارهای پیشین انجام‌شده در این زمینه می‌پردازیم.

%2019
در سال ۲۰۱۹ پژوهش \cite{li2019diagnosis} در باب این مسئله چاپ شد.
هدف اصلی این پژوهش، تشخیص سرطان تیروئید با استفاد از تصاویر سونوگرافیک و مبتنی بر روش‌های شبکه‌های عصبی پیچشی عمیق\LTRfootnote{Deep Convolutional Neural Network - DCNN} بوده است. ورودی مدل، تصاویر معمولی سونوگرافی از غده و خروجی مدل وجود یا عدم وجود ناحیه‌های سرطانی بوده ‌است.

%2019
در همان سال در پژوهش آزمایشی \cite{guan2019deep} گوان و همکاران، هدف خود را ابتدا طبقه‌بندی سرطان تیروئید به سه گروه فولیکولار ادنوما\LTRfootnote{Follicular Adenoma} و فولیکولار کارسینوما\LTRfootnote{Follicular Carcinoma} و پاپیلاری کارسینوما\LTRfootnote{Papillary Carcinoma} قرار داده و در مرحله بعد سعی کردند دسته فولیکولار کارسینوما را به چهار زیر‌گروه دیگر طبقه‌بندی کنند. این تیم، از اسلاید‌های سیتولوژی دیجیتال استفاده کردند و نتایج خود را بر روی هر دو مدل \lr{VGG16} و \lr{InceptionV3} گزارش کردند.
در این پژوهش، از اسلاید‌ها، تصاویر کوچک‌تری استخراج شد و از این تصاویر برای آموزش مدل‌ها استفاده شد. 
در نهایت بهترین نتیجه را با مدل \lr{VGG16} و با دقت 97.66 برروی تصاویر استخراج‌شده کسب کردند.

باز در همین سال، مقاله \cite{nguyen2019artificial} منتشر شد که هدف آن تشخیص بد‌خیم و یا خوشخیم بودن غده تیروئید بوده است.
در این پژوهش از تصاویر سونوگرافی استفاده کردند، اما اینبار هر دو دامنه فضایی عکس و دامنه فوریه عکس را مورد بررسی قرار دادند.
معماری سیستم توسعه داده شده بدین صورت است که، تصاویر ابتدا وارد طبقه‌بند مبتنی بر فوریه می شوند.
در این قسمت توزیع مولفه‌های فویر مورد بررسی قرار می گیرد و با توجه به آن پیش‌طبقه‌بندی بر روی تصویر انجام می شود.
درنهایت خروجی قسمت قبل، وارد شبکه عمیقی می‌شود که طبقه‌بندی نهایی را انجام می دهد.
شبکه عمیق به‌کار رفته در این قسمت از نوع \lr{ResNet} تغییر یافته بوده که جزییات بیشتری در مورد معماری آن در فصل‌های بعدی آمده‌است.

%2020
در پژوهش دیگر \cite{elliott2020application} که در سال 2020 میلادی منتشر شد، دنیل و همکاران خود سیستمی را توسعه دادند تا معیار \lr{TBSR} را از روی اسلاید های دیجیتال پیشبینی کند. این تیم در ابتدا با استفاده از شبکه \lr{VGG11} مدلی را آموزش دادند تا ناحیه های مورد نظر\LTRfootnote{Region Of Interest} یک اسلاید را تشخیص دهد. برای این کار اسلایدها را به صورت تصادفی به ناحیه های مختلف تقسیم کرند و مدل را بر روی این تصاویر آموزش دادند.
سپس باز با استفاده از شبکه \lr{VGG11} شبکه‌ی مستقل دیگری را آموزش دادند تا با به‌کارگیری ناحیه های مورد نظر بدست آمده، \lr{TBSR} نمونه را پیشبینی کند.



%2021
در سال ۲۰۲۱ نیز در پژوهش \cite{bohland2021machine} سعی بر این شد مدلی برای شخیص سرطان تیروئید از نوع پاپیلاری توسعه داده شود.
در این مسیر، روش‌های مختلف و مدل‌های مختلفی مورد بررسی قرارگرفتند و مقایسه‌ای از عملکرد هر یک گزارش شد.
دو روشی که در این پژوهش مورد بررسی قرار گرفتند به صورت زیر هستند.
\begin{itemize}
    \item طبقه‌بندی بر اساس ویژگی\newline
    قدم اول این روش، مرزبندی\LTRfootnote{Segmentation} است که در آن، هدف مشخص کردن ناحیه‌های مربوط به هسته و سلول است.
    برای بدست آوردن مرزبندی مناسب از شبکه \lr{UNet} استفاده شد. این شبکه ابتدا در سال ۲۰۱۵ و در مقاله \cite{ronneberger2015u} معرفی شد.
    این شبکه، یک معماری رمزگذار-رمزگشای U شکل است که از چهار بلوک رمزگذار و چهار بلوک رمزگشا تشکیل شده است.
    پل‌هایی بین رمزگذار و رمزگشا وجود دارد که آن‌ها را به هم متصل می کند و جریان اطلاعات را کامل می کند.
    این اتصالات منجر به الحاق ویژگی های متفاوت از نظر معنایی می شود که عملکرد شبکه را برای مرزبندی بهبود می دهد.
    برای فرآیند استخراج ویژگی، تصاویر تیروئید، مرزبندی‌های مرتبط و ماسک‌های تهیه‌شده از سلول‌ها استفاده شده است تا ویژگی‌های مختلف سلول بدست بیایند.
    در مرحله بعد، ویژگی‌های بدست آمده جمع می‌شوند و روی آن‌ها پیش‌پردازش انجام می‌گیرد تا مورد استفاده مدل برای طبقه‌بندی قرار گیرند. الگوریتم‌های متفاوت و کلاسیکی مانند طبقه‌بندی بردار پشتیبانی، نزدیک‌ترین همسایه، بیز ساده گوسی و ... برای این کار مورد آزمایش قرار گرفت که در نهایت بهترین آن‌ها انتخاب گردید.

    \item طبقه‌بندی مبتنی بر یادگیری عمیق\newline
    این روش، روش مستقیمی برای طبقه‌بندی به حساب می‌آید، به همین دلیل نیاز به استخرج ویژگی به صورت دستی نیست و قدم‌ها مورد نیاز برای طبقه‌بندی به شدت کاهش پیدا می‌کنند.
    باید توجه شود که در روش اول، از یادگیری عمیق تنها برای مرزبندی تصاویر استفاده شد و برای طیقه‌بندی نهایی، الگوریتم‌های کلاسیک به‌کار گرفته شدند، از این جهت پیچیدگی مرزبندی روش اول به تنهایی با پیچیدگی روش دوم برابری می کند.
    برای کاهش زمان آموزش مدل‌ها نیز، از مدل‌های از پیش آموزش داده‌شده بر روی مجموع‌داده \lr{ImageNet}\cite{deng2009imagenet} استفاده شد. در نهایت قبل از استفاده از داده‌ها، روش‌های مختلف از جمله چرخش تصویر، آیینه کردن، متعادل سازی هسیتوگرارم تصویر\LTRfootnote{Contrast Limited Adaptive Histogram Equalization}، تطبیق دامنه فوریه و ... برای داده‌افزایی به‌کار گرفته شد.
     

\end{itemize}
بهترین نتیجه‌ای که توسط این پژوهش گزارش شد، از روش مبتنی بر ویژگی و با دقت‌های 83 و 89 بر روی دو مجموع‌داده بدست آمد.
در هر دو مجموع‌داده دقت روش مبتنی بر ویژگی بالاتر از روش یادگیری عمیق بوده، اما در پایان، پیشنهاد شد که برای مجموع‌داده‌های کوچک تر و مرزی تر(منظور از مرزی بودن داده در اینجا این است که به سادگی نمی توان به گروه‌های مورد نظر طبقه‌بندی کرد) از روش مبتنی بر ویژگی و برای مجموع‌داده‌های بزرگ تر و کلی تر از روش مبتنی بر یادگیری عمیق استفاده گردد.