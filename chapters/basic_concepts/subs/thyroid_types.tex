\subsection{انواع سرطان تیروئید}\label{subsec:انواع-سرطان-تیروئید}

سرطان تیروئید انواع مختلفی دارد که در هر کدام، ویژگی‌ها سلول متفاوت است.

\begin{itemize}
    \item سرطان پاپیلاری\LTRfootnote{Papillary Carcinomas}:
    این سرطان در ناحیه‌های کوچکی از بدن پخش می شود و حدود 80 درصد از نمونه سرطان‌های تیروئید رو تشکیل می دهد.
    \item سرطان فولیکیولار\LTRfootnote{Follicular Carcinomas}:
    این نوع از سرطان در ناحریه بزرگتری از بدن و در زمان کمتری پخش می شود و حدود 15 درصد از نمونه سرطان‌های تیروئید رو تشکیل می دهد.
    \item سرطان مدولار\LTRfootnote{Medullary carcinoma}:
    این نوع از سرطان حدود 3 درصد از جمعیت کل بیماران سرطان غده تیروئید را تشکیل می دهد.
    \item سرطان‌های نادر دیگر
\end{itemize}

با توجه به سطح خطری که سرطان‌ها برای بیماران دارند نیز می توان آن‌ها را دسته بندی کرد.
این دسته بندی مربوط به نوع سرطان تیروئید نیست و می توان سرطان‌های دیگری را نیز با این معیار دسته بندی کرد.
\begin{itemize}
    \item سرطان بدخیم\LTRfootnote{Benign Carcinomas}:
    این سرطان در ناحیه‌های کوچکی از بدن پخش می شود و حدود 80 درصد از نمونه سرطان‌های تیروئید رو تشکیل می دهد.
    \item سرطان خوشخیم\LTRfootnote{Malignant Carcinomas}:
    این نوع از سرطان در ناحریه بزرگتری از بدن و در زمان کمتری پخش می شود و حدود 15 درصد از نمونه سرطان‌های تیروئید رو تشکیل می دهد.
\end{itemize}

\newpage