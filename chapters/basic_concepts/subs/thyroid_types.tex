\subsection{انواع سرطان تیروئید}\label{subsec:انواع-سرطان-تیروئید}

سرطان تیروئید انواع مختلفی دارد که در هر کدام، ویژگی‌ها سلول متفاوت است.

\begin{itemize}
    \item سرطان پاپیلاری\LTRfootnote{Papillary Carcinomas}:
    این سرطان در ناحیه‌های کوچکی از بدن پخش می‌شود و حدود 80 درصد از نمونه سرطان‌های تیروئید رو تشکیل می‌دهد و در سلول‌های فولیکولی تولید کننده تیروگلوبولین در تیروئید شروع می‌شود.
    \item سرطان فولیکیولار\LTRfootnote{Follicular Carcinomas}:
    این نوع از سرطان در ناحیه بزرگتری از بدن و در زمان کمتری پخش می‌شود و حدود 15 درصد از نمونه سرطان‌های تیروئید رو تشکیل می‌دهد.
    \item سرطان مدولار\LTRfootnote{Medullary Carcinoma}:
    این نوع از سرطان حدود 3 درصد از جمعیت کل بیماران سرطان غده تیروئید را تشکیل می‌دهد که از سلول‌های تولید کننده کلسی تونین در تیروئید ناشی می‌شود.
    \item انواع نادر دیگر
\end{itemize}

هر کدام از انوع سرطان اشاره شده در قسمت اخیر را می‌توان با توجه به سطح خطری که آن‌ها برای بیماران دارند نیز دسته‌بندی کرد.
این دسته‌بندی مربوط به نوع سرطان تیروئید نیست و می‌توان سرطان‌های دیگری را نیز با این معیار دسته‌بندی کرد.
\begin{itemize}
    \item سرطان بدخیم\LTRfootnote{Malignant Carcinomas}:
    تومور بدخیم دارای مرزهای نامنظم است، سریعتر از یک تومور خوش‌خیم رشد می‌کند و می‌تواند به سایر قسمت‌های بدن شما نیز سرایت کند.
    \item سرطان خوش خیم\LTRfootnote{Benign Carcinomas}:
    تومور خوش‌خیم دارای مرزهای مشخص، صاف و منظم است، یک تومور خوش‌خیم می‌تواند بسیار بزرگ شود اما به بافت مجاور حمله نمی‌کند.
\end{itemize}