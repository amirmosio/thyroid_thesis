\subsection{روش‌های تشخیص سرطان تیروئید}\label{subsec:روش-های-تشخیص-سرطان-تیروئید}

برای تشخیص این سرطان روش‌های بسیاری وجود دارد.
از جمله این روش‌ها می‌توان به موارد زیر اشاره کرد.

\begin{itemize}
    \item معاینه فیزیکی:
    در این روش پزشک با معاینه بیمار به صورت فیزیکی و بررسی ناحیه تیروئید ممکن است متوجه غده سرطانی شود.
    \item سی تی اسکن\LTRfootnote{Computed tomography (CT) scan}:
    این آزمایش با استفاده از پرتوهای اشعه ایکس و به منظور تشخیص سایز و میزان انتشار تومورهای سرطانی تیروئید در سایر بخش‌های بدن انجام می‌شود.
    به طور مثال، اگر فردی مشکوک به سرطان تیروئید باشد، سی تی‌اسکن از ناحیه گردن و قفسه سینه یا قسمت دیگری از بدن مانند شکم نیز انجام می‌شود.
    \item بیوپسی\LTRfootnote{Biopsy}:
    در صورت وجود توده در ناحیه گردن ، پزشک با استفاده از يک سوزن نازک ، اقدام به نمونه برداری از بافت توده می‌کند تا وجود احتمالی سرطان را با استفاده از انجام آزمایشات روی این نمونه تشخیص دهد.
    \item روش‌های دیگر از جمله آزمایش‌های خون و ژنتیک، سونوگرافی گردن، اسکن رادیو یودین\LTRfootnote{Radioiodine scans} و \ldots
\end{itemize}

همانطور که اشاره شد، یکی از روش‌های تشخیص سرطان تیروئید، بیوپسی است.
در ادامه به جزییات بیشتری از این روش پرداخته می‌شود زیرا در مرحله اول روش استاندارد طلایی تشخیص این سرطان است و همچنین در این پژوهش، تمرکز کلی و ارایه روش‌های هوش مصنوعی مبتنی بر این روش خواهند بود.

\subsubsection{تشخیص سرطان تیروئید به روش بیوپسی}
روش اصلی تشخیص این سرطان بیوپسی است.
در صورتی که پزشک تشخیص دهد که بیمار نیاز به بیوپسی دارد، اسان‌ترین راه برای تشخیص سرطانی بودن توده،
نمونه برداری سوزنی\LTRfootnote{Fine Needle Aspiration (FNA)} است.
این نوع بیوپسی در بعضی مواقع می‌تواند، در کلینیک و دفتر پزشک انجام شود.

پزشک یک سوزن نازک و توخالی را مستقیماً در غده قرار می‌دهد تا تعدادی سلول و چند قطره مایع را داخل سرنگ بریزد.
دکتر ممکن است این فرآیند را دو یا سه بار تکرار کند تا از مکان‌های مختلف غده نمونه برداری کند.
نمونه‌های گرفته شده، سپس در زیر میکروسکوپ توسط متخصصین مورد بررسی قرار می‌گیرد تا از روی ویژگی‌های سلول‌های سرطانی از جمله شکل، اندازه، رنگ، هسته سلول، ساختار سلول‌ها و ... فرآیند تشخیص کامل شود.

برای راحتی کار یا انجام فرآیند بررسی یه صورت ریموت، می‌توان این نمونه‌ها را توسط دستگاه‌هایی اسکن کرد تا اسلاید دیجیتالی از آن‌ها ایجاد شود.
به این روش تصویر برداری تمام لغزشی\LTRfootnote{Whole-Slide Imaging} می‌گویند.
از مزایای دیگر این روش این است که اسلاید‌ها به صورت فایل‌های دیجیتال در می‌آیند، که می‌توانند به اشتراک گذاشته شوند و در فرآیند‌های آموزشی و تحصیلی و حتی یادگیری ماشین مورد استفاده قرار گیرند.

