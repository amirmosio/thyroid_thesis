\فصل{مقدمه}

یادگیری ماشین و به خصوص پردازش تصویر امروزه یکی از پر کاربردترین تکنولوژی های روز دنیا است که در حوزه های مختلف صنعتی، پزشکی، شهری، امنیتی، علمی و فنی کاربرد فراوانی دارد.
در واقع پردازش تصویر به پردازش تصاویر آنالوگ و دیجیتالی می پردازد که به ترتیب توسط دوربین های معمولی و ابزارهای مانند اسکنر نوری، دوربین و یا حسگر دیجیتالی گرفته شده اند
که امروزه بیشتر به موضوع پردازش تصویر دیجیتال گفته می‌شود که شاخه‌ای از پردازش سیگنال است که با پردازش سیگنال دیجیتال که نماینده تصاویر برداشته شده با دوربین دیجیتال یا اسکن شده توسط اسکنر هستند سر و کار دارد.

در این پژوهش سعی شده با استفاده از این روش و با استفاده از اسلاید های دیجیتال ثبت شده از بیمار که شامل سلول های غده تیروئید هستند، روشی برای تشخیص سریع تر و دقیق تر سرطان تیروئید ارائه شود.

\section{تاریخچه}\label{sec:تاریخچه}

\section{سرطان تیروئید}\label{sec:سرطان-تیروئید}
غده تیروئید یک غده پروانه ای شکل است که در قسمت تحتانی گردن قرار دارد.
این غده وظیفه کنترل متابولیسم و سوخت و ساز بدن را به عهده دارد.
علاوه بر این، تیروئید هورمون هایی تولید می کند که وظیفه تنظیم دمای بدن ، میزان سوخت و ساز و مصرف اکسیژن را انجام می دهد.
سرطان تیروئید زمانی رخ می دهد که سلول های غده تیروئید بر اثر عواملی،  دچار تغییر می شوند و بعد از آن سلول های دارای ناهنجاری شروع به تکثیر می کنند و به تدریج تومور تیروئید را تشکیل می دهند.
سرطان تیروئید در صورتی که زودهنگام تشخیص داده شود یکی از قابل درمان ترین انواع سرطان است.

\subsection{انواع سرطان تیروئید}\label{subsec:انواع-سرطان-تیروئید}

سرطان تیروئید انواع مختلفی دارد که در هر کدام، ویژگی‌ها سلول متفاوت است.

\begin{itemize}
    \item سرطان پاپیلاری\LTRfootnote{Papillary Carcinomas}:
    این سرطان در ناحیه‌های کوچکی از بدن پخش می‌شود و حدود 80 درصد از نمونه سرطان‌های تیروئید رو تشکیل می‌دهد و در سلول‌های فولیکولی تولید کننده تیروگلوبولین در تیروئید شروع می‌شود.
    \item سرطان فولیکیولار\LTRfootnote{Follicular Carcinomas}:
    این نوع از سرطان در ناحیه بزرگتری از بدن و در زمان کمتری پخش می‌شود و حدود 15 درصد از نمونه سرطان‌های تیروئید رو تشکیل می‌دهد.
    \item سرطان مدولار\LTRfootnote{Medullary Carcinoma}:
    این نوع از سرطان حدود 3 درصد از جمعیت کل بیماران سرطان غده تیروئید را تشکیل می‌دهد که از سلول‌های تولید کننده کلسی تونین در تیروئید ناشی می‌شود.
    \item انواع نادر دیگر
\end{itemize}

هر کدام از انوع سرطان اشاره شده در قسمت اخیر را می‌توان با توجه به سطح خطری که آن‌ها برای بیماران دارند نیز دسته‌بندی کرد.
این دسته‌بندی مربوط به نوع سرطان تیروئید نیست و می‌توان سرطان‌های دیگری را نیز با این معیار دسته‌بندی کرد.
\begin{itemize}
    \item سرطان بدخیم\LTRfootnote{Malignant Carcinomas}:
    تومور بدخیم دارای مرزهای نامنظم است، سریعتر از یک تومور خوش‌خیم رشد می‌کند و می‌تواند به سایر قسمت‌های بدن شما نیز سرایت کند.
    \item سرطان خوش خیم\LTRfootnote{Benign Carcinomas}:
    تومور خوش‌خیم دارای مرزهای مشخص، صاف و منظم است، یک تومور خوش‌خیم می‌تواند بسیار بزرگ شود اما به بافت مجاور حمله نمی‌کند.
\end{itemize}
\subsection{روش های تشخیص سرطان تیروئید}\label{subsec:روش-های-تشخیص-سرطان-تیروئید}

برای تشخیص این سرطان روش های بسیاری وجود دارد.
از جمله این روش ها می توان به موارد زیر اشاره کرد.

\begin{itemize}
    \item معاینه فیزیکی\LTRfootnote{Physical Exam}:
    در این روش پزشک با معاینه بیمار به صورت فیزیکی و بررسی ناحیه تیروئید ممکن است متوجه غده سرطانی شود.
    \item سی تی اسکن\LTRfootnote{Computed tomography (CT) scan}:
    این آزمایش با استفاده از پرتوهای اشعه ایکس و به منظور تشخیص سایز و میزان انتشار تومورهای سرطانی تیروئید در سایر بخش های بدن انجام می شود.
    \item بیوپسی\LTRfootnote{Biopsy}:
    در صورت وجود توده در ناحیه گردن ، پزشک با استفاده از يک سوزن نازک ، اقدام به نمونه برداری از بافت توده می کند تا وجود احتمالی سرطان را با استفاده از انجام آزمایشات روی این نمونه تشخیص دهد.
    \item روش های دیگر از جمله آزمایش های خون\LTRfootnote{Blood Test} و ژنتیک،\LTRfootnote{Genetic Test} سونوگرافی گردن، اسکن رادیو یودین\LTRfootnote{Radioiodine scans} و \ldots
\end{itemize}

همانطور که اشاره شد، یکی از روش های تشخیص سرطان تیروئید، بیوپسی است.
در ادامه می خواهم به جزییات بیشتری از این روش بپردازم زیرا در این پژوهش، تمرکز کلی و ارایه روش های هوش مصنوعی مبنی بر این روش است.

\subsubsection{تشخیص سرطان تیروئید به روش بیوپسی}
روش اصلی تشخیص این سرطان بیوپسی است.
به این صورت که سلول ها از ناحیه های مشکوک به سرطان خارج می شوند و سپس در آزمایشگاه مورد بررسی قرار می گیرند.

در صورتی که پزشک تشخیص دهد که بیمار نیاز به بیوپسی دارد، اسان ترین راه برای تشخیص سرطانی بودن توده،
نمونه برداری سوزنی\LTRfootnote{Fine Needle Aspiration (FNA)} است.
این نوع بیوپسی در بعضی مواقع می تواند، در کلینیک و دفتر پزشک انجام شود.

قبل از انجام نمونه برداری ممکن است بی حس کننده موضعی مورد استفاده قرار گیرد ولی در بیشتر موارد، به بی حسی نیازی نیست.
بعد از آن پزشک یک سوزن نازک و توخالی را مستقیماً در غده قرار می دهد تا تعدادی سلول و چند قطره مایع را داخل سرنگ بریزد.
دکتر ممکن است این فرآیند را دو یا سه بار تکرار کند تا از مکان های مختلف غده نمونه برداری کند.
نمونه های گرفته شده، سپس در زیر میکروسکوپ توسط متخصصین مورد بررسی قرار می گیرد تا از روی ویژگی های سلول های سرطانی از جمله شکل، اندازه، رنگ، هسته سلول، ساختار سلول ها و ... فرآیند تشخیص کامل شود.

برای راحتی کار یا انجام فرآیند بررسی یه صورت ریموت، می توان این نمونه ها را توسط دستگاه هایی اسکن کرد تا اسلاید دیجیتالی از آن ها ایجاد شود.
به این روش تصویر برداری تمام لغزشی\LTRfootnote{Whole-Slide Imaging} می گویند.
از مزایای دیگر این روش این است که اسلاید ها به صورت فایل های دیجیتال در می آیند، که می توانند به اشتراک گذاشته شوند و در فرآیند های آموزشی و تحصیلی و حتی یادگیری ماشین مورد استفاده قرار گیرند.


\newpage

\subsection{اسلاید های دیجیتال}\label{subsec:اسلاید-های-دیجیتال}
همانطور که در قسمت قبل گفته شد، نمونه های تهییه شده از ناحیه های مشکوک بیمار از غده تیروئید، توسط دستگاه هایی اسکن می شوند و به صورت تصاویری دیجیتال در می آیند که برای کامپیوتر های قابل خواندن هستند.
قبل از اینکه این نمونه ها توسط دستگاه اسکن شوند ممکن است فرآیندی روی نمونه ها انجام شود تا سلول ها در زیر میکروسکپ و یا در تصویر دیجیتال به خود رنگ بگیرند.
این کار به متخصصین کمک می کنند تا سلول ها را راحت تر از دیگر مواد داخل نمونه تشخیص دهند و این باعث افزایش دقت تشخیص می شود.
به همین دلیل اسلاید های تهییه شده به این روش، ممکن است رنگ های مختلفی به خود بگیرند.


نمونه ای از این اسلاید ها در شکل ~\autoref{fig:sampleWSIscan} آمده است.
\begin{figure}
    \begin{center}
        \includegraphics[width=\linewidth]{figs/Sample_WSI.PNG}
    \end{center}
    \caption{نمونه اسلاید اسکن شده از غده تیروئید}
    \label{fig:sampleWSIscan}
\end{figure}
این اسلاید ها معمولا ابعاد بسیار بزرگی دارند و ممکن است تا بزرگنمایی 40 برابر را پشتیبانی کنند.
به همین دلیل ممکن است حجمی بین چند مگابایت تا چند گیگابایت را به خود بگیرند.

مدل هوش مصنوعی که در این پژوهش توسعه داده می شود، باید قادر باشد ویژگی های سلول ها را از روی این اسلاید ها تشخیص داده و تخمین درستی از وضعیت سرطان به ما بدهد.



\section{یادگیری ماشین و پردازش تصویر}\label{sec:یادگیری-ماشین-و-پردازش-تصویر}

\section{تعریف مسئله}\label{sec:تعریف-مسئله2}\label{sec:تعریف-مسئله}

