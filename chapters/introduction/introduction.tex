\فصل{مقدمه}

یادگیری ماشین و به خصوص پردازش تصویر امروزه یکی از پر کاربردترین تکنولوژی‌های روز دنیا است که در حوزه‌های مختلف صنعتی، پزشکی، شهری، امنیتی، علمی و فنی کاربرد فراوانی دارد.
در واقع پردازش تصویر امروزه بیشتر به موضوع پردازش تصویر دیجیتال گفته می‌شود که شاخه‌ای از پردازش سیگنال است که با پردازش سیگنال دیجیتال که نماینده تصاویر برداشته شده با دوربین دیجیتال یا اسکن شده توسط اسکنر هستند سر و کار دارد.

تشخیص سرطان با استفاده از پردازش تصویر یکی از موضوعات تحقیقاتی است که از مدت‌هاپیش مورد توجه پژوهشگران بوده است. تشخیص سرطان به معنی تشخیص وجود سلول‌های سرطانی در نمونه مورد بررسی است که ویژگی‌های متفاوتی از سلول‌های عادی دارند. در گذشته راهکارهای ارائه شده مبتنی بر پیدا کردن ویژگی‌های سلولی بوده که در نهایت سیستم تلاش به پیدا کردن این ویژگی‌ها می کرده است اما با روی کار آمدن شبکه‌های عمیق، نتایج بهتری بدست آمد.

\section{تعریف مسئله}\label{sec:تعریف-مسئله}
تجزیه و تحلیل بافت هیستوپاتولوژی استاندارد بسیارخوبی در تشخیص سرطان در نظر گرفته می شود. تصویربرداری کل اسلاید\LTRfootnote{Whole Slide Image}، به عنوان مثال، اسکن و دیجیتالی کردن کل اسلایدهای بافت شناسی، اکنون در سراسر جهان در آزمایشگاه‌های آسیب شناسی مورد استفاده قرار می گیرد. هیستوپاتولوژیست‌های آموزش دیده می‌توانند تشخیص دقیق نمونه‌های بیوپسی را بر اساس داده‌های اسلاید‌های کل ارائه دهند. با توجه به ابعاد اسلاید‌ها و افزایش تعداد موارد سرطان بالقوه، تجزیه و تحلیل این تصاویر فرآیندی زمان بر است. تقسیم‌بندی خودکار بافت تومور به افزایش دقت، سرعت و تکرارپذیری تحقیق کمک می‌کند. در گذشته نه چندان دور، تکنیک‌های مبتنی بر یادگیری عمیق، نتایج پیشرفته‌ای را در طیف گسترده‌ای از وظایف تحلیل تصویر، از جمله تجزیه و تحلیل اسلایدهای دیجیتالی، ارائه کرده‌اند. با این حال، راه‌حل‌های مبتنی بر یادگیری عمیق، چالش‌های بسیاری از جمله نبود داده عمومی زیاد، اندازه بزرگ داده‌های اسلاید‌های کل، ناهمگونی در تصاویر و پیچیدگی ویژگی‌ها را به همراه دارند.

در این پژوهش سعی شده است که راه حلی  برای تشخیص سرطان تیرویید از روی اسلاید‌های کل سایتولوژی پیشنهاد شود.
مراحل کار به این صوت است که، در ابتدا دیتاست‌هایی معرفی می شوند. این داده‌ها به روش‌های مختلفی و از مکان‌های مختلفی جمع‌آوری شده‌اند، به همین دلیل ممکن است تنوع و توزیع داده‌ها متفاوت باشد.
در نهایت از این داده‌ها برای آموزش و ارزیابی مدل‌های مبتنی بر شبکه‌های عمیق استفاده می شود و روش‌های مختلفی بکار گرفته شود که عملکرد مدل را بر روی این مجموع دادگان بهبود ببخشد.

\section{اهمیت موضوع}\label{sec:اهمیت موضوع}
سرطان تیرویید یکی از سرطان‌های شایع بین انسان‌ها است که تشخیص سریع و زودهنگام آن بسیار در درمان غده موثر است.
همانطور که پیشتر نیز گفته شد، فرآیند تشخیص سرطان از روی اسلاید‌های کل، به دلیل اندازه بزرگ این اسلایدها، فرآیندی بسیار وقت گیر است، از این روی، استفاده از فرآیند‌های اتوماتیک مبتنی بر یادگیری ماشین که دقت بالایی را نشان دهند، از هدر رفت وقت متخصصان جلوگیری می کند و از سوی دیگر فرآیند تشخیص را تسریع می بخشند.


\section{ساختار پایان‌نامه}\label{sec:ساختار پایان‌نامه}
این پایان نامه شامل شش فصل است که در فصل اول به مقدمه پرداخته شد.
فصل دوم شامل مفاهیم اولیه مورد نیاز برای مسئله و راه حل پیشنهادی ارائه شده است.
در فصل سوم به کارهای پیشین که در این زمینه انجام گرفته است، اشاره شده است.
در فصل چهارم، مجموع دادگان مورد بررسی معرفی می شوند.
در فصل پنجم چالش ها، پیش پردازش داده ها، شبکه های عمیق استفاده شده و روش های بهبود مورد بحث قرار می گیرند و در نهایت مطالب بیان شده در فصل ششم جمع بندی می شوند.
 


