\فصل{مقدمه}

یادگیری ماشین و به خصوص پردازش تصویر امروزه یکی از پر کاربردترین تکنولوژی‌های روز دنیا است که در حوزه‌های مختلف صنعتی، پزشکی، شهری، امنیتی، علمی و فنی کاربرد فراوانی دارد.
پردازش تصویر امروزه بیشتر به موضوع پردازش تصویر دیجیتال گفته می‌شود که شاخه‌ای از پردازش سیگنال است که با پردازش سیگنال دیجیتال که نماینده تصاویر برداشته‌شده با دوربین دیجیتال یا اسکن‌شده توسط اسکنر هستند سر و کار دارد.

تشخیص سرطان تیروئید با استفاده از پردازش تصویر یکی از موضوعات تحقیقاتی است که از مدت‌ها پیش مورد توجه پژوهشگران بوده است. تشخیص این سرطان به معنی تشخیص وجود سلول‌های سرطانی در نمونه مورد بررسی است که ویژگی‌های متفاوتی از سلول‌های عادی دارند.

قبل از ادامه کار باید با دو کلمه در ادبیات پزشکی مربوط به سرطان آشنا شد.
بیوپسی\LTRfootnote{Biopsy} یک آزمایش پزشکی است که معمولا توسط جراح، رادیولوژیست یا متخصص قلب و عروق انجام می‌شود.
این فرآیند شامل استخراج سلول‌ها یا بافت‌های نمونه برای بررسی و تعیین وجود یا وسعت بیماری است. کلمه‌ی دیگر، پتالوژی\LTRfootnote{Pathology} علم علل و عوارض بیماری‌ها، به شاخه‌ای از پزشکی می‌گویند که به بررسی آزمایشگاهی نمونه‌های بافت بدن برای اهداف تشخیصی یا پزشکی قانونی می‌پردازد.
یکی از راه‌های تشخیصی سرطان تیروئید، بیوپسی غده و بعد از آن بررسی نمونه توسط پتالوژیست\LTRfootnote{Pathologist} است. پزشک متخصص نمونه را در زیر میکروسکپ مورد بررسی قرار می‌دهد و ناحیه‌های مشکوک را تشخیص می‌دهد. با توجه به اندازه نمونه و اسلاید‌ها، این فرآیند ممکن است ساعت‌ها به طول بیانجامد و فرآیندیست که به دقت بالایی نیازمند است.
تصویربرداری کل اسلاید\LTRfootnote{Whole Slide Image}، به عنوان مثال، اسکن و دیجیتالی کردن اسلایدهای بافت شناسی، اکنون در سراسر جهان در آزمایشگاه‌های آسیب شناسی مورد استفاده قرار می‌گیرد.
در گذشته راهکارهای ارائه‌شده برای حل این مسئله توسط کامپیوتر‌ها و با استفاده از تصویربرداری کل اسلاید، مبتنی بر روش‌های کلاسیک طبقه‌بندی و پیدا کردن ویژگی‌های سلولی به صورت دستی بوده است، اما با روی کار آمدن شبکه‌های عمیق، نتایج بهتری قابل کسب است که به دقت تشخیص و تسریع آن کمک می‌کند.

\section{تعریف مسئله}\label{sec:تعریف مسئله}
در این پژوهش، هدف ارائه‌ی راه‌حلی برای تشخیص سرطان تیروئید و طبقه‌بندی وضعیت آن از روی اسلاید‌های دیجیتال کل سایتولوژی است.
به این صورت است که مجموع‌داده‌هایی از اسلاید‌های دیجیتال معرفی می‌شوند. سپس از این داده‌ها برای آموزش و ارزیابی مدل‌های مبتنی بر شبکه‌های عمیق برای طبقه‌بندی وضعیت تیروئید استفاده می‌شود. طبقه‌بندی‌هایی که در این پژوهش مورد بررسی قرار می‌گیرند در قسمت اول بدخیم و خوشخیم بودن غده تیروئید است و بعد از آن نیز مدلی برای تشخیص سلول‌های توموری توسعه داده می‌شود.

\section{چالش‌های مسئله}\label{sec:چالش‌های مسئله}
تقسیم‌بندی خودکار بافت تومور به افزایش دقت، سرعت و تکرارپذیری تحقیق کمک می‌کند. در گذشته نه چندان دور، تکنیک‌های مبتنی بر یادگیری عمیق، نتایج پیشرفته‌ای را در طیف گسترده‌ای از وظایف تحلیل تصویر، از جمله تجزیه و تحلیل اسلایدهای دیجیتالی، ارائه کرده‌اند. با این حال، راه‌حل‌های مبتنی بر یادگیری عمیق، چالش‌های بسیاری از جمله، نبود داده عمومی کافی، اندازه‌ی بزرگ اسلاید‌های کل، ناهمگونی در تصاویر و پیچیدگی ویژگی‌ها را به همراه دارند.



\section{اهمیت موضوع}\label{sec:اهمیت موضوع}
سرطان تیروئید یکی از سرطان‌های شایع بین انسان‌ها است که تشخیص سریع و زودهنگام آن بسیار در درمان غده موثر است.
همانطور که پیشتر نیز گفته شد، فرآیند تشخیص سرطان از روی اسلاید‌های کل، به دلیل اندازه بزرگ این اسلایدها، فرآیندی بسیار وقت‌گیر است، از این روی، استفاده از فرآیند‌های خودکار مبتنی بر یادگیری ماشین که دقت بالایی را نشان دهند، از هدر رفت وقت متخصصان جلوگیری می‌کند و از سوی دیگر فرآیند تشخیص را تسریع می‌بخشند.


\section{ساختار پایان‌نامه}\label{sec:ساختار پایان‌نامه}
در فصل اول این پایان نامه به مسئله و مقدمات آن پرداخته شد.
فصل دوم شامل مفاهیم اولیه مورد نیاز برای مسئله و راه حل پیشنهادی ارائه شده است.
در فصل سوم به کارهای پیشین که در این زمینه انجام گرفته است، اشاره شده است.
در فصل چهارم، مجموع دادگان، چالش‌ها، پیش‌پردازش داده‌ها، شبکه‌های عمیق استفاده‌شده و روش‌های بهبود مورد بحث قرار می‌گیرند و در آخر مطالب بیان‌شده در فصل پنجم جمع‌بندی شده‌است.
 


