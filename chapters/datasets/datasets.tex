\فصل{مجموع‌داده‌ها}

یکی از چالش‌هایی که در حل این مسئله با آن روبرو هستیم، نبود مجموع‌داده مناسب و با دسترسی عمومی است و گام اول برای این پژوهش به حساب می آید.
در سال 2010 مقاله \cite{halicek2019head} منتشر شد که هدف آن تشخیص سرطان با به‌کارگیری اسلاید‌های ناحیه‌های سر و گردن بوده است.
این پژوهش، بعد از آموزش مدل بر روی مجموع‌داده خصوصی تهیه شده، سعی شد تا برای ارزیابی خارجی، مدل را بر روی مجموع‌داده عمومی \lr{CAMELYON 2016} تست کنند. این مجموع‌داده در چالشی در سال 2016 منتشر شد که شرکت کنندگان باید از اآن برای حل مسئله استفاده می‌کردند. هدف اصلی این چالش تشخیص سرطان سینه بوده و مجموع‌داده نیز شامل اسلاید‌ها و برچسب‌هایی مربوط به این سرطان بوده است.
در مقاله‌ای دیگر \cite{bohland2021machine} مجموع‌داده عمومی از اسلاید‌های دیجیتال و برچسب‌های آن تهیه شد. هدف آن تشخیص پاپیلاری بودن یا نبودن غده تیروئید بوده‌است که علیرغم اعلام نویسنده، موفق به تماس برای دریافت مجموع‌داده نشدیم.
با توجه به گفته‌های اخیر، جمع آوری داده یکی از بخش‌های مهم این پروژه است تا با آن بتوان مدل را به خوبی آموزش داد و عملکرد خوبی را دریافت کرد.
در ادامه به مجموع‌داده‌هایی که تهیه شد و درنهایت برای آموزش و ارزیابی مدل به‌کار گرفته شد اشاره می‌کنیم.

\subsection{اطلس ژنوم سرطان تیروئید}\label{subsec:موسسه-ملی-سرطان-پورتال-داده-های-مشترک-ژنومیک}

اطلس ژنوم سرطان تیروئید\LTRfootnote{The Cancer Genome Atlas-Thyroid Cancer/TCGA-THCA} بخشی از یک تلاش بزرگتر برای ایجاد یک جامعه تحقیقاتی متمرکز بر اتصال فنوتیپ‌های\LTRfootnote{Phenotypes} سرطان به ژنوتیپ‌ها\LTRfootnote{Genotypes} با ارائه تصاویر بالینی منطبق با افراد از اطلس ژنوم سرطان (TCGA) است.\cite{clark2013cancer}

\lr{Cite: "The results <published or shown> here are in whole or part based upon data generated by the TCGA Research Network: https://www.cancer.gov/tcga."}
شناسه پروژه به صورت \lr{TCGA-THCA} است که در آدرس \cite{ncigdc} قابل مشاهده است. این پروژه شامل اطلاعات 507 بیماراست که هدف اصلی آن، بررسی سرطان تیرویید در این بیماران است.

این اطلاعات شامل توالی دی ان ای، جنسیت، سن، داده‌های کلینیکال از بررسی بیمار توسط پزشک، اسلایدهای دیجیتال نمونه برداری شده از غده تیرویید و ... است.
تعداد کل اسلاید‌های جمع آوری شده مورد استفاده ما از این پروژه 639 عدد است که در مجموع حجمی بالغ بر 105 گیگابایت دارند. برای هرکدام از این اسلاید‌ها سه برچسب درصد سلول‌های عادی، درصد سلول‌های سرطانی و درصدر سلول‌ها استورمال آمده است که ما نیز از این برچسب‌ها برای مسئله خود استفاده می کنیم. از بین این اسلاید‌ها و با توجه به درصد سلول‌ها نرمال، سرطانی و استورمال 106 اسلاید صددرصد سرطانی و 76 اسلاید صددرصد عادی برای ادامه کار انتخاب شد.
در مرحله بعد برای استفاده از این 182 اسلاید، آن ها، به 265418 قطعه عکس کوچک تر در آورده شدند تا بعد از آن بتوانند در آموزش مدل مورد استفاده قرار بگیرند. 
\subsection{پاپ سوسایتی اطلس تیرویید}\label{subsec:پاپ-سوسایتی}
پاپ سوسایتی، انجمنی است که به عنوان یک شرکت اداره می شود و منحصراً به عنوان یک مؤسسه خیریه معاف از مالیات برای سازمان های علمی آموزشی، مدنی و خیریه سازماندهی شده است.
اطلس تیرویید این انجمن\cite{papsocietyiamgeatlas}، شامل نزدیک به 300 تصویر از غده تیرویید و نواحی مختلف آن از اسمیر مستقیم یک نمونه برداری سوزنی است که تمامی تصاویر به طور مستقل توسط اعضای کمیته بررسی و تایید شده است.
تصاویر توسط اسکریپتی از صفحات وب این اطلس جمع آوری شد. تصاویر این مجموعه داده حجمی حدود 125 ماگابایت دارند که از بین این تصاویر، 269 عکس به عنوان، بدخیم و خوشخیم بچسب زده شده اند. در ادامه از تصاویر این دو گروه، نزدیک به 400 تصویر کوچک تر استخراج شد تا مناسب برای آموزش مدل باشند و مورد استفاده ارزیابی مدل قرار بگیرند. 
\subsection{پایگاه داده ریزآرایه بافت استانفورد}\label{subsec:پایگاه-داده-ریزآرایه-بافت-استانفورد}
