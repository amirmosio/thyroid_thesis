\فصل{فصل}
\section{دیتاست ها}
غده تیروئید یک غده پروانه ای شکل است که در قسمت تحتانی گردن قرار دارد.
این غده وظیفه کنترل متابولیسم و سوخت و ساز بدن را به عهده دارد.
علاوه بر این، تیروئید هورمون هایی تولید می کند که وظیفه تنظیم دمای بدن ، میزان سوخت و ساز و مصرف اکسیژن را انجام می دهد.
سرطان تیروئید زمانی رخ می دهد که سلول های غده تیروئید بر اثر عواملی، دچار تغییر می شوند و بعد از آن سلول های دارای ناهنجاری شروع به تکثیر می کنند و به تدریج تومور تیروئید را تشکیل می دهند.
سرطان تیروئید در صورتی که زودهنگام تشخیص داده شود یکی از قابل درمان ترین انواع سرطان است.

\subsection{موسسه ملی سرطان پورتال داده های مشترک ژنومیک}\label{subsec:موسسه-ملی-سرطان-پورتال-داده-های-مشترک-ژنومیک}

\section{پاپ سوسایتی اطلس تیروئید}\label{subsec:پاپ-سوسایتی}
پاپ سوسایتی، انجمنی است که به عنوان یک شرکت اداره می‌شود و منحصراً به عنوان یک مؤسسه خیریه معاف از مالیات برای سازمان‌های علمی آموزشی، مدنی و خیریه سازماندهی شده است.
اطلس تیروئید این انجمن\cite{papsocietyiamgeatlas}، شامل نزدیک به 300 تصویر از غده تیروئید و نواحی مختلف آن از اسمیر مستقیم یک نمونه برداری سوزنی است که تمامی تصاویر به طور مستقل توسط متخصصین بررسی و برچسب گذاری شده‌اند.
تصاویر توسط اسکریپتی از صفحات وب این اطلس جمع آوری شد و که هر کدام از تصاویر ابعادی بین
 $1752\times1485$
 و
 $525\times360$
 را دارا هستند.
مجموع تصاویر این مجموع‌داده حجمی $125$ ماگابایتی دارند که از بین این تصاویر، $269$ عکس به عنوان، بدخیم و خوشخیم بچسب زده شده‌اند.
نمونه‌هایی از تصاویر این مجموع‌داده در \ref{fig:samplecoloringfrompapsociety}  و در فصل قبل آمده است.
%در انتها، از تصاویر این دو گروه، $396$ تصویر کوچک تر برای استفاده استخراج شد.
\subsection{پایگاه داده ریزآرایه بافت استانفورد}\label{subsec:پایگاه-داده-ریزآرایه-بافت-استانفورد}
