\فصل{دیتاست‌ها}

همانطور که در \cite{hosseini2019atlas} نیز اشاره شده است، یکی از مشکلاتی که در حل این مسائل با آن روبرو هستیم، نبود دیتاست مناسب و با دسترسی عمومی است. در این مقاله حسینی و دیگر اعضای تیم سعی کردند تا دیتاست عمومی از روی اسلاید‌های دیجیتال شامل ارگان‌های مختلف بدن تهییه کنند.

در سال 2010 \cite{halicek2019head} نیز که هدف نهایی آن تشخیص سرطان با اسلاید‌های ناحیه‌های سر و گردن است، بعد از آموزش مدل بر روی دیتاست خصوصی تهیه شده، سعی شد تا برای ارزیابی خارجی، مدل را بر روی دیتاست عمومی \lr{CAMELYON 2016} که تست کنند. دیتاست \lr{CAMELYON 2016} در چالشی در سال 2016 منتشر شد که شرکت کنندگان باید از این دیتاست برای حل مسئله استفاده می کردند. هدف اصلی این چالش تشخیص سرطان سینه بوده و دیتاست نیز شامل اسلاید‌ها و برچسب‌هایی مربوط به این سرطان بوده است.

در مقاله ای دیگر \cite{bohland2021machine} دیتاست عمومی از اسلاید‌های دیجیتال و برچسب‌های آن تهیه شد که هدف آن تشخیص پاپیلاری بودن یا نبودن غده تیروئید بوده‌است درحالی که موفق به تماس برای دریافت دیتاست نشدیم.

با توجه به گفته‌های اخیر، جمع آوری داده یکی از بخش‌های مهم این پروژه است تا با آن بتوان مدل را به خوبی آموزش داد و عملکرد خوبی را دریافت کرد.
در ادامه به دیتاست‌هایی که تهییه شد و درنهایت برای آموزش و ارزیابی مدل بکار گرفته شد اشاره می کنیم.

\subsection{موسسه ملی سرطان پورتال داده های مشترک ژنومیک}\label{subsec:موسسه-ملی-سرطان-پورتال-داده-های-مشترک-ژنومیک}

\section{پاپ سوسایتی اطلس تیروئید}\label{subsec:پاپ-سوسایتی}
پاپ سوسایتی، انجمنی است که به عنوان یک شرکت اداره می‌شود و منحصراً به عنوان یک مؤسسه خیریه معاف از مالیات برای سازمان‌های علمی آموزشی، مدنی و خیریه سازماندهی شده است.
اطلس تیروئید این انجمن\cite{papsocietyiamgeatlas}، شامل نزدیک به 300 تصویر از غده تیروئید و نواحی مختلف آن از اسمیر مستقیم یک نمونه برداری سوزنی است که تمامی تصاویر به طور مستقل توسط متخصصین بررسی و برچسب گذاری شده‌اند.
تصاویر توسط اسکریپتی از صفحات وب این اطلس جمع آوری شد و که هر کدام از تصاویر ابعادی بین
 $1752\times1485$
 و
 $525\times360$
 را دارا هستند.
مجموع تصاویر این مجموع‌داده حجمی $125$ ماگابایتی دارند که از بین این تصاویر، $269$ عکس به عنوان، بدخیم و خوشخیم بچسب زده شده‌اند.
نمونه‌هایی از تصاویر این مجموع‌داده در \ref{fig:samplecoloringfrompapsociety}  و در فصل قبل آمده است.
%در انتها، از تصاویر این دو گروه، $396$ تصویر کوچک تر برای استفاده استخراج شد.
\subsection{پایگاه داده ریزآرایه بافت استانفورد}\label{subsec:پایگاه-داده-ریزآرایه-بافت-استانفورد}
