\فصل{نتیجه گیری}
در اکثر مسائل یادگیری ماشین، توزیع دادگان و تعمیم پذیر بودن مدل روی توزیع داده‌های مشابه، بسیار مهم است. 

همانطور که پیشتر نیز گفته شد یکی از مشکلات اصلی که در این پژوهش با آن روبرو هستیم، نبود مجموع دادگان عمومی کافی است.
در این پژوهش ابتدا سعی شد مجموع داده‌های متفاوتی جمع آوری و معرفی شوند.
در مرحله بعد مدل‌های متفاوتی بر روی این دادگان آموزش داده شد و با وجود چالش‌هایی نظیر حجم عظیم داده‌ها، فیلتر کردن ناحیه‌های پس زمینه و ... با به‌کار بردن روش‌های پیش‌پردازشی و تنظیم ابرپارامتر‌ها، عملکرد خوبی بدست آمد.

سه مجموع‌داده استفاده‌شده در این پژوهش توزیع‌های بسیار متفاوتی دارند و برای اهداف متفاوتی مورد استفاده قرار گرفتند.
دو مجموع‌داده اطلس پاپسوسایتی و ریزآرایه استنفورد برای طبقه‌بندی بدخیم و خوش خیم بودن تصاویر نمونه مورد استفاده قرار گرفتند که بهترین دقتی که بر روی آن‌ها بدست آمد به ترتیب 94.98 بوده است. تصاویر موجود در این دو مجموع‌داده، حالت مطالعه موردی\LTRfootnote{Case Study} داشته و از تنوع زیادی در نحوه اسکن، اندازه اسلاید، روش رنگ آمیزی شیمیایی سلول‌ها و ... برخوردار بوده است، از این جهت آموزش مدل بر روی این تصاویر دشوار بوده است.
مجموع‌داده بزرگ تر دیگری که در آموزش مدل به‌کار رفت، اطلس ژنوم سرطان تیروئید بوده است که وجود یا عدم وجود سلول‌های توموری در بهترین حالت با دقت  $99.72$ بر روی تصاویر تشخیص داده شد.

برای ادامه کار می‌توان تمرکز بیشتری بر روی توزیع‌های متفاوت دادگان گذاشته شود و با استفاده بیشتر از مفاهیمی چون تطبیق دامنه و یا انتقال یادگیری\LTRfootnote{ُTransfer Learning} قابلیت تعمیم پذیر بودن مدل بروی دادگان متفاوت و برای وظایف متفاوت بیشتر شود.