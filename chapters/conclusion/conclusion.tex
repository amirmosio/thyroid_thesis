\فصل{جمع‌بندی}

همانطور که پیشتر نیز گفته شد یکی از مشکلات اصلی که در این پژوهش با آن روبرو هستیم، نبود مجموع دادگان عمومی کافی است.
در این پژوهش ابتدا سعی شد مجموع داده‌های متفاوتی جمع آوری و معرفی شوند.
در مرحله بعد مدل‌های متفاوتی بر روی این دادگان آموزش داده شد و با وجود چالش‌هایی نظیر حجم عظیم داده‌ها، فیلتر کردن ناحیه‌های پس‌زمینه و دیگر موارد، با به‌کار بردن روش‌های پیش‌پردازشی و تنظیم ابرپارامتر‌ها، عملکرد خوبی بدست آمد.

نتایج نهایی مدل‌ها، ارقام قابل قبولی را برای دقت آن‌ها اعلام می‌کنند. با این حال، برای استفاده از آن‌ها و مفید واقع شدن آنها در دنیای امروزی، نیاز به کار بیشتر است. به عنوان مثال، برای اطمینان از عملکرد مدل‌ها می‌توان از مجموع داده‌های خارجی استفاده کرد که ارزیابی خارجی، معیار مناسبی و اطمینان بیشتری برای عملکرد کلی مدل است. جدای از این موضوع، برای ادامه کار و پیشبرد عملکرد مدل‌ها، تمرکز بیشتری را می‌‌تان بر روی توزیع‌های متفاوت دادگان گذاشت، که استفاده بیشتر از مفاهیمی چون تطبیق دامنه و یا انتقال یادگیری\LTRfootnote{ُTransfer Learning} را می‌طلبد.